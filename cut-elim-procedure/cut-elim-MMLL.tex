
\documentclass[12pt]{article}
 
\usepackage[margin=0.8in]{geometry} 
\usepackage{packages}
\usepackage{ amsmath,amsthm,amssymb, hyperref, bussproofs}
 \usepackage{stmaryrd}
 \usepackage{proof}
 \usepackage{multicol, multirow}
 \usepackage{cmll}
 \usepackage{hyperref}
 \usepackage{tcolorbox}
 \usepackage{pdflscape}
\usepackage{tikz}
\usetikzlibrary{matrix}
\usepackage{mycommands}
\usepackage{cancel}
 
\begin{document}
 
\title{
    {\large \textbf{Cut Elimination in LL with Multimodalities}}}%
\author{Bruno Xavier \& Carlos Olarte}
 
\maketitle

\begin{abstract}
    This document contains all the needed proof transformation to prove
    cut-admissibility for the focused system for linear logic with multimodalities.
\end{abstract}
\section{The system and the cut rules}
The rules of the system are in Figure \ref{system:ALLF}.
The cut-elimination procedure requires the cut-rules
presented below. In the following section, it is shown how
the are (mutually) eliminated. 


%!TEX root = main.tex
% !TEX spellcheck = en-US
\begin{figure}
	\hrule
	\vspace{.15cm}
	
	\noindent
\resizebox{\textwidth}{!}{	
	$
	\small
	\begin{aligned}[c]
	\textbf{Negative rules:\qquad}
	\end{aligned}
	\begin{aligned}[c]
	\begin{array}{c}
   \infer[\top]{\unfoc{\Theta}{\Gamma}{\top, L}}{}
  \qquad
	\infer[\bot]{\unfoc{\Theta}{\Gamma}{\bot, L}}{\unfoc{\Theta}{\Gamma}{L}}
	\qquad
	\infer[\lpar]{\unfoc{\Theta}{\Gamma}{F\lpar G, L}}{\unfoc{\Theta}{\Gamma}{F,G, L}}\qquad
	\infer[\store]{\unfoc{\Theta}{\Gamma}{S, L}}{\unfoc{\Theta}{\Gamma,S}{L}}
	\\\\
	\infer[\with]{\unfoc{\Theta}{\Gamma}{ F \with G, L}}
	{\unfoc{\Theta}{\Gamma}{ F, L} &
		\unfoc{\Theta}{\Gamma}{ G, L}}
	\qquad
	\infer[\forall]{\unfoc{\Theta}{\Gamma}{\forall x. F, L}}{\unfoc{\Theta}{\Gamma}{ F[y/x], L}}
	\qquad
	\infer[\store_\mathsf{s}]{\unfoc{\Theta}{\Gamma}{\nquest{i} F, L}}{\unfoc{\Theta,i:F}{\Gamma}{L}}
	\end{array}
	\end{aligned}
	$}
	
\vspace{.3cm}
\hrule
\vspace{.3cm}
	\resizebox{\textwidth}{!}{
	$
	\begin{aligned}[c]
	\textbf{Positive rules:\qquad}
	\end{aligned}
	\begin{aligned}[c]
	\begin{array}{c}
	\infer[\tensor]{\foc{\Theta^u,\Theta_1^l, \Theta_2^l}{\Gamma_1,\Gamma_2}{F \tensor G}}{\foc{\Theta^u,\Theta_1^l}{\Gamma_1}{ F} &
		 \foc{\Theta^u,\Theta_2^l}{\Gamma_2}{ G}}
	\quad
	\infer[\oplus_1]{\foc{\Theta}{\Gamma}{F_1\oplus F_2}}{\foc{\Theta}{\Gamma}{F_1}} 
	\quad 
	\infer[\oplus_2]{\foc{\Theta}{\Gamma}{F_1\oplus F_2}}{\foc{\Theta}{\Gamma}{F_2}} 
	\\\\
%		\infer[\one]{\mathcal{E}\lns[i]\foc{\Psi}{\Gamma}{\one}}{}
\quad
	\infer[\exists]{\foc{\Theta}{\Gamma}{\exists x. F}}{\foc{\Theta}{\Gamma}{F[t/x]}} 
%	\qquad
%	\infer[\quest]{\unfoc{\Theta,G}{\Gamma}{\cdot}~\lnse\unfoc{\Upsilon}{\cdot}{F}}{ \unfoc{\Theta}{\Gamma}{\cdot}~\lnse\unfoc{\Upsilon,G}{\cdot}{F}}
%	\qquad
%	\infer[\bang]{\Escr\lns\foc{\Theta}{\Gamma}{\bang F}}{\unfoc{\Theta}{\Gamma}{\cdot}\,\lnse \unfoc{\cdot}\cdot{F}}
		\quad
		   \infer[\one]{\foc{\Theta}{\cdot}{\one}}{}
%	\infer[\top]{\mathcal{E}\lns\foc{\Theta}{\Gamma}{\top}}{}
\end{array}
\end{aligned}
	$
	}
	\vspace{.3cm}
\hrule
\vspace{.3cm}

	
	\resizebox{\textwidth}{!}{
	$
	\small
	\begin{aligned}[c]
	\textbf{Structural:\quad} 
	\end{aligned}
	\begin{aligned}[c]
	\begin{array}{c}
	\infer[\mathsf{I_l}]{\foc{\Theta^u}{A}{A^\perp}}{}
	\qquad
	\infer[\mathsf{l_s}]{\foc{\Theta^u, i:A}{\cdot}{A^\perp}}{}
	\\\\
%	\infer[\mathsf{I_2}]{\mathcal{G}\lns[i]\foc{\Theta,A}{\Gamma}{A^\perp}}{}
%	\\\\
	\infer[\mathsf{D_l}]{\unfoc{\Theta}{\Gamma, P}{\cdot}}{\foc{\Theta}{\Gamma}{P}}
	\qquad
	\infer[\mathsf{D^u_s}]{\unfoc{\Theta,i:P_a}{\Gamma}{\cdot}}{\foc{\Theta,i:P_a}{\Gamma}{P_a}}  
	\quad
		\infer[\mathsf{D^l_s}]{\unfoc{\Theta,i:P_a}{\Gamma}{\cdot}}{\foc{\Theta}{\Gamma}{P_a}}  
	\qquad
%	\infer[\mathsf{R_{\re}}]{\Gscr~\lns[i]e\unfoc{\Theta}{\cdot}{F}}{\Gscr\lns[i]\unfoc{\Theta}{\cdot}{F}}
%	\qquad
	\infer[\mathsf{R_n}]{\foc{\Theta}{\Gamma}{N}}{\unfoc{\Theta}{\Gamma}{N}}\\
	\end{array}
	\end{aligned}
	$
	}
	\vspace{.3cm}
\hrule
\vspace{.3cm}

\resizebox{\textwidth}{!}{
$
	\small
	\begin{aligned}[c]
	\textbf{Modal:} 
	\end{aligned}
	\begin{aligned}[c]
	\begin{array}{c}
	\infer[\nbang{i}]{\foc{\Theta}{\cdot}{\nbang{i} F}}{\unfoc{\Theta}{\cdot}{\cdot}\,\lnse[i] \unfoc{\cdot}\cdot{F}} 
\qquad
\infer[\mathsf{R_{\re}}]{\unfoc{\Theta^u}{\cdot}{\cdot}\lnse[i]\unfoc{\Upsilon}{\cdot}{L}}{\unfoc{\Upsilon}{\cdot}{L}}
\qquad
\infer[{\mathsf{D_d}}]{\unfoc{\Theta}{\cdot}{\cdot}}{\unfoc{\Theta}{\cdot}{\cdot}\,\lnse[i] \unfoc{\cdot}\cdot{\cdot}}   
\qquad
\infer[\mathsf{\nbang{\lclassic}{}}]{\foc{\Theta^u}{\cdot}{\nbang{\lclassic}{F}}}{\unfoc{\Theta^u}{\cdot}{{F}}}
\\\\
\infer[\nquest{i}_{\4}]{\unfoc{\Theta,j:F}{\Gamma}{\cdot}~\lnse[i]\unfoc{\Upsilon}{\cdot}{L}}{ \unfoc{\Theta}{\Gamma}{\cdot}~\lnse[i]\unfoc{\Upsilon,j+:F}{\cdot}{L}}
\quad
	\infer[\nquest{i}_{\mathsf{kl}}]{\unfoc{\Theta,j:F}{\cdot}{\cdot}~\lnse[i]\unfoc{\Upsilon}{\cdot}{L}}{ \unfoc{\Theta}{\cdot}{\cdot}~\lnse[i]\unfoc{\Upsilon}{\cdot}{L,F}}
	\quad
		\infer[\nquest{i}_{\mathsf{ku}}]{\unfoc{\Theta,j:F}{\cdot}{\cdot}~\lnse[i]\unfoc{\Upsilon}{\cdot}{L}}{ \unfoc{\Theta}{\cdot}{\cdot}~\lnse[i]\unfoc{\Upsilon,\lclassic:F}{\cdot}{L}}

	\end{array}
	\end{aligned}
$}\\
	\hrule
	\caption{End-active focused system $\LNS_\FALL$. 
	$\Theta^u$ (resp. $\Theta^l$) contains only unbounded (resp. linear) subexponentials.
	In  $\mathsf{l_s}$ and $\mathsf{I_l}$, $A$ is atomic. In $\forall$, $y$ is fresh. In $\store$, $S$ is a literal or a positive formula. In $\mathsf{\mathsf{R_n}}$, $N$ is a negative formula.
	In $\mathsf{D_l}$, $P$ is positive, and in $\mathsf{D^u_s}, \mathsf{D^l_s}$, $P_a$ is not atomic.   
In $\mathsf{D^u_s}, \mathsf{D^l_s}$ and $\mathsf{l_s}$,  $\T \in \mathcal{U}(i)$.  
In all question-marked rules   $i\preceq j$. Moreover, 
$i\neq \lclassic$ in $\nbang{i}{}$; 
 $\D\in \mathcal{U}(i)$ in $\mathsf{D_d}$; 
 $\4 \in \mathcal{U}(j)$ in $?^i_\4$;
  $\{\4,\C,\W\} \cap \mathcal{U}(j) = \emptyset$ in $?^i_\mathsf{kl}$;
  $\4 \not\in \mathcal{U}(i)$ and 
  $\U \subseteq \mathcal{U}(i)$ in 
  $?^i_\mathsf{ku}$ and in $\mathsf{D^u_s}$. 
%$\Escr$ is either an empty list or a sequent of the form $\unfoc{\Theta}{\cdot}{\cdot}$. 
}\label{system:ALLF}
%\vspace{-0.065cm}
\end{figure}   

 
  	\begin{center}
 		$
 		\begin{array}{ccc ccc}
 			
 			\infer[cut_1]{\unfoc{\Kcal}{\Gamma,\Delta}{L_1,L_2}}{
 				\deduce{\unfoc{\Kcal_1}{\Gamma,\cutF}{L_1}}{}
 				&
 				\deduce{\unfoc{\Kcal_2}{\Delta}{\cutF^\perp,L_2}}{}
 			}
 			\quad
 			\infer[cut_2]{\unfoc{\Kcal}{\Gamma,\Delta}{L}}{
 				\deduce{\unfoc{\Kcal_1}{\Gamma}{\cutF,L}}{}
 				&
 				\deduce{\foc{\Kcal_2}{\Delta}{\cutF^\perp}}{}
 			}
 			\\\\
 			\infer[\cut_3]{\unfoc{\Kcal}{\Gamma}{L}}{
 				\deduce{\unfoc{\Kcal_1,\cutF_i}{\Gamma}{L}}{}
 				&
 				\deduce{\foc{\Kcal_2}{\cdot}{\nbang{i}\cutF^\perp}}{}
 			}
 			\\\\
 			\infer[cut_4]{\unfoc{\Kcal}{\Gamma}{F}}{
 				\deduce{\foc{\Kcal_1,\cutF_i}{\Gamma}{F}}{}
 				&
 				\deduce{\foc{\Kcal_2}{\cdot}{\nbang{i} \cutF^\perp}}{}
 			}
 			\qquad
 			\infer[\cut_5]{\unfoc{\Kcal}{\Gamma,\Delta}{L_1,L_2,L_3}}{
 				\deduce{\unfoc{\Kcal_1}{\Gamma}{L_1,\cutF,L_2}}{}
 				&
 				\deduce{\unfoc{\Kcal_2}{\Delta}{\cutF^\perp,L_3}}{}
 			}
 			\qquad
 		\end{array}$
 		
 	\end{center}		
 	
 	Where $\cutF$ is the cut-formula, $\fF$ is a formula, $\Gamma$ and $\Delta$ are multisets of formulas, $\Kcal_1$ and $\Kcal_2$ are indexed contexts of formulas, and $\fL$ is a list of formulas. 
 
    Moreover, $\Kcal_1$, $\Kcal_2$ and $\Kcal$ have the same unbounded formulas and the linear formulas in $\Kcal$ are the multiset union of the linear formulas in $\Kcal_1$ and $\Kcal_2$.  
    
\newpage
\section{Exponential Rules}
{\footnotesize
	\begin{multicols}{2}[]
		{	
			
			\begin{tcolorbox}
				$i \preceq j$, $4 \notin \mathcal{U}(j)$ and $j$ is unbounded 
				\begin{prooftree}
					\AxiomC{}
					\noLine
					\UnaryInfC{$\Theta;\cdot\Uparrow\cdot\lns{i}\fCenter \Upsilon,\textcolor{red}{F_{c}}; \cdot\Uparrow L $}
					\RightLabel{[$\wn_{Ku}$]}
					\UnaryInfC{$\Theta, \textcolor{red}{F_j};\cdot\Uparrow\cdot\lns{i}\fCenter \Upsilon ; \cdot\Uparrow L $}
				\end{prooftree}
			\end{tcolorbox}
			
			\begin{tcolorbox}
				$i \preceq j$, $4 \notin \mathcal{U}(j)$ and $j$ is linear
				\begin{prooftree}
					\AxiomC{}
					\noLine
					\UnaryInfC{$\Theta;\cdot\Uparrow\cdot\lns{i}\fCenter \Upsilon; \cdot\Uparrow L,\textcolor{red}{F} $}
					\RightLabel{[$\wn_{Kl}$]}
					\UnaryInfC{$\Theta, \textcolor{red}{F_j};\cdot\Uparrow\cdot\lns{i}\fCenter \Upsilon ; \cdot\Uparrow L $}
				\end{prooftree}
			\end{tcolorbox}
		}
	\end{multicols}
	
	\begin{tcolorbox}
		$i \preceq j$ and $4 \in \mathcal{U}(j)$
		\begin{prooftree}
			\AxiomC{}
			\noLine
			\UnaryInfC{$\Theta;\cdot\Uparrow\cdot\lns{i}\fCenter \Upsilon,\textcolor{red}{F_{j+}}; \cdot\Uparrow L $}
			\RightLabel{[$\wn_{K4}$] }
			\UnaryInfC{$\Theta, \textcolor{red}{F_j};\cdot\Uparrow\cdot\lns{i}\fCenter \Upsilon ; \cdot\Uparrow L $}
		\end{prooftree}
	\end{tcolorbox}
	
	\begin{multicols}{2}[]
		{
			\begin{tcolorbox}
				$\Theta$ is unbounded and $\forall j \in\Theta, i \npreceq j$
				\begin{prooftree}
					\AxiomC{}
					\noLine
					\UnaryInfC{$\fCenter \Upsilon ; \cdot\Uparrow L $}
					\RightLabel{[$R_{r}$]}
					\UnaryInfC{$\Theta;\cdot\Uparrow\cdot\lns{i}\fCenter \Upsilon ; \cdot\Uparrow L $}
				\end{prooftree}
			\end{tcolorbox}
			\begin{tcolorbox}
				$\Theta$ is unbounded
				\begin{prooftree}
					\AxiomC{}
					\noLine
					\UnaryInfC{$\fCenter \Theta ; \cdot\Uparrow P $}
					\RightLabel{[$\nbang{c}$]}
					\UnaryInfC{$\fCenter \Theta ; \cdot\Downarrow \nbang{lc} P $}
				\end{prooftree}
			\end{tcolorbox}
		}
	\end{multicols}
	
	\begin{multicols}{2}[]
		{
			\begin{tcolorbox}
				
				\begin{prooftree}
					\AxiomC{}
					\noLine
					\UnaryInfC{$\Theta;\cdot\Uparrow\cdot\lns{i}\fCenter \cdot ; \cdot\Uparrow F $}
					\RightLabel{[$\nbang{i}$]}
					\UnaryInfC{$\Theta;\cdot\Downarrow\nbang{i} F$}
				\end{prooftree}
			\end{tcolorbox}
			\begin{tcolorbox}
				\begin{prooftree}
					\AxiomC{}
					\noLine
					\UnaryInfC{$\Theta;\cdot\Uparrow\cdot\lns{i}\fCenter \cdot ; \cdot\Uparrow\cdot$}
					\RightLabel{[$D \in \mathcal{U}(i)$]}
					\UnaryInfC{$\Theta;\cdot\Uparrow\cdot$}
				\end{prooftree}
			\end{tcolorbox}
		}
	\end{multicols}
}

\begin{itemize}
	\item The subexponential $c$ is not related with anyone
	\item $\mathcal{U}(c)=\{K,U,T\}$
	\item Let $\Theta$ be a subexp. context, $\mathcal{F}(\Theta)$ is all formulas in $\Theta$ 
	\item Let $\Theta$ be a subexp context, $\Theta^u$ (resp. $\Theta^l$) is the unbounded (resp. linear) context in $\Theta$
	\item Note that $\Theta\equiv\Theta^u,\Theta^l$ with $\Theta^u\cap\Theta^l=\emptyset$
	\item  $\Upsilon_{_a\preceq}$ means that $\forall b:subexp, b \in \Upsilon \implies a \preceq b$
	
	\item  $\Upsilon^Z$ means that $\forall i:subexp, i \in \Upsilon \implies Z\in\mathcal{U}(i)$ 
	\item  $\Upsilon^{\bcancel{Z}}$ means that $\forall i:subexp, i \in \Upsilon \implies Z\notin\mathcal{U}(i)$ 

\end{itemize}


\section{Height-Preserving Lemmas}

\begin{theorem}[$\W$]
If $U\in\mathcal{U}(j)$ and	$\fCenter\Theta;\Delta;\Updownarrow X$ then $\fCenter F_j,\Theta;\Delta;\Updownarrow X$
\end{theorem}

\begin{theorem}[$\C$]
	If $U\in\mathcal{U}(j)$ and	$\fCenter F_j,F_j,\Theta;\Delta;\Updownarrow X$ then $\fCenter F_j,\Theta;\Delta;\Updownarrow X$
\end{theorem}

\begin{theorem}[$\C_c$]
	If $\{U,T\}\subseteq\mathcal{U}(j)$ and	$\fCenter F_c,F_j,\Theta;\Delta;\Updownarrow X$ then $\fCenter F_j,\Theta;\Delta;\Updownarrow X$
\end{theorem}

\begin{theorem}[$\A_c$]
	If $\{U,T\}\subseteq\mathcal{U}(j)$ and	$\fCenter F_j,\Theta;F,\Delta;\Updownarrow X$ then $\fCenter F_j,\Theta;\Delta;\Updownarrow X$
\end{theorem}


\begin{theorem}[$\A_l$]
	If $T\in\mathcal{U}(j)$ and	$\fCenter \Theta;F,\Delta;\Updownarrow X$ then $\fCenter F_j,\Theta;\Delta;\Updownarrow X$
\end{theorem}

\newpage
\section{Elimination of $\cut_1$}
{\footnotesize	
	\vspace{0.3cm}	
\begin{center}
	$	\infer[cut_1]{\unfoc{\Kcal}{\Gamma,\Delta}{L_1,L_2}}{
	\deduce{\unfoc{\Kcal_1}{\Gamma,\cutF}{L_1}}{}
	&
	\deduce{\unfoc{\Kcal_2}{\Delta}{\cutF^\perp,L_2}}{}
}$
\end{center}
	\vspace{0.3cm}	
}



{\footnotesize

		\vspace{0.3cm}
	\begin{minipage}{0.4\textwidth}
		\begin{prooftree}
			\AxiomC{}
			\noLine
			\UnaryInf$\fCenter\Kcal_1 : \Gamma,\cutF \Uparrow  L_1$
			\UnaryInf$\fCenter \Kcal_1 : \Gamma,\cutF \Uparrow  \bot, L_1$
			\AxiomC{}
			\noLine
			\UnaryInfC{$\Pi_2$}
			\BinaryInfC{$\vdash \Kcal : \Gamma,\Delta\Uparrow \bot, L_1,L_2$}
		\end{prooftree}
	\end{minipage}
	\begin{minipage}{0.1\textwidth}
		\begin{center}
			$\rightsquigarrow$
		\end{center}
	\end{minipage}
	\begin{minipage}{0.3\textwidth}
		\begin{prooftree}
			\AxiomC{}
			\noLine
			\UnaryInfC{$\vdash \Kcal_1 : \Gamma,\cutF \Uparrow  L_1$}
			\AxiomC{}
			\noLine
			\UnaryInfC{$\Pi_2$}
			\BinaryInf$\fCenter \Kcal : \Gamma,\Delta\Uparrow L_1,L_2$
			\UnaryInf$\fCenter \Kcal : \Gamma,\Delta\Uparrow \bot, L_1,L_2$
		\end{prooftree}
	\end{minipage}
	\vspace{0.3cm}

	
	\vspace{0.3cm}
	\begin{minipage}{0.4\textwidth}
		\begin{prooftree}
			\AxiomC{}
			\noLine
			\UnaryInf$\fCenter \Kcal_1 : \Gamma,\cutF \Uparrow  P, Q, L_1$
			\UnaryInf$\fCenter \Kcal_1 : \Gamma,\cutF \Uparrow  P\parr Q, L_1$
			\AxiomC{}
			\noLine
			\UnaryInfC{$\Pi_2$}
			\BinaryInfC{$\vdash \Kcal : \Gamma,\Delta\Uparrow P\parr Q, L_1,L_2$}
		\end{prooftree}
	\end{minipage}
	\begin{minipage}{0.1\textwidth}
		\begin{center}
			$\rightsquigarrow$
		\end{center}
	\end{minipage}
	\begin{minipage}{0.3\textwidth}
		\begin{prooftree}
			\AxiomC{}
			\noLine
			\UnaryInf$\fCenter \Kcal_1 : \Gamma,\cutF \Uparrow  P, Q, L_1$
			\AxiomC{}
			\noLine
			\UnaryInfC{$\Pi_2$}
			\BinaryInf$\fCenter \Kcal : \Gamma,\Delta\Uparrow P, Q, L_1,L_2$
			\UnaryInf$\fCenter \Kcal : \Gamma,\Delta\Uparrow P\parr Q, L_1,L_2  $
		\end{prooftree}
	\end{minipage}
	\vspace{0.3cm}
	
	\vspace{0.3cm}	
	\begin{minipage}{0.4\textwidth}
		\begin{prooftree}
			\AxiomC{}
			\noLine
			\UnaryInfC{$\vdash \Kcal_1 :\Gamma,\cutF \Uparrow  P, L_1$}
			\AxiomC{}
			\noLine
			\UnaryInfC{$\vdash \Kcal_1 : \Gamma,\cutF \Uparrow  Q, L_1$}
			\BinaryInf$\fCenter \Kcal_1 : \Gamma,\cutF \Uparrow  P\with Q, L_1$
			\AxiomC{}
			\noLine
			\UnaryInfC{$\Pi_2$}
			\BinaryInfC{$\vdash \Kcal : \Gamma,\Delta\Uparrow P\with Q, L_1,L_2$}
		\end{prooftree}
	\end{minipage}

	\begin{minipage}{0.3\textwidth}
		\begin{center}
			$\rightsquigarrow$
		\end{center}
	\end{minipage}
	\begin{minipage}{0.3\textwidth}
		\begin{prooftree}
			\AxiomC{}
			\noLine
			\UnaryInf$\fCenter \Kcal_1 :  \Gamma,\cutF \Uparrow  P, L_1$
			\AxiomC{}
			\noLine
			\UnaryInfC{$\Pi_2$}
			\BinaryInfC{$\vdash \Kcal : \Gamma,\Delta \Uparrow  P, L_1,L_2$}
			\AxiomC{}
			\noLine
			\UnaryInf$\fCenter \Kcal_1 : \Gamma,,\cutF \Uparrow  Q, L_1$
			\AxiomC{}
			\noLine
			\UnaryInfC{$\Pi_2$}
			\BinaryInfC{$\vdash \Kcal : \Gamma,\Delta \Uparrow  Q, L_1,L_2$}
			\BinaryInfC{$\vdash \Kcal : \Gamma,\Delta\Uparrow P\with Q, L_1,L_2$}
		\end{prooftree}
	\end{minipage} 
	\vspace{0.3cm}
	
	\vspace{0.3cm}
	\begin{minipage}{0.4\textwidth}
		\begin{prooftree}
			\AxiomC{}
			\noLine
			\UnaryInf$\fCenter \Kcal_1 : \Gamma,\cutF \Uparrow  P[c/x], L_1$
			\UnaryInf$\fCenter \Kcal_1 : \Gamma,\cutF \Uparrow  \forall xP,L_1$
			\AxiomC{}
			\noLine
			\UnaryInfC{$\Pi_2$}
			\BinaryInfC{$\vdash \Kcal : \Gamma,\Delta\Uparrow \forall xP,L_1,L_2$}
		\end{prooftree}
	\end{minipage}
	\begin{minipage}{0.1\textwidth}
		\begin{center}
			$\rightsquigarrow$
		\end{center}
	\end{minipage}
	\begin{minipage}{0.3\textwidth}
		\begin{prooftree}
			\AxiomC{}
			\noLine
			\UnaryInfC{$\vdash \Kcal_1 : \Gamma,\cutF \Uparrow P[c/x], L_1$}
			\AxiomC{}
			\noLine
			\UnaryInfC{$\Pi_2$}
			\BinaryInfC{$\vdash \Kcal : \Gamma,\Delta\Uparrow  P[c/x], L_1,L_2$}
			\UnaryInfC{$\vdash \Kcal: \Gamma,\Delta\Uparrow \forall xP, L_1,L_2$}
		\end{prooftree}
	\end{minipage} 
	\vspace{0.3cm}

	\vspace{0.3cm}	
\begin{minipage}{0.4\textwidth}
	\begin{prooftree}
		\AxiomC{}
		\noLine
		\UnaryInf$\fCenter\Kcal_1,P_i: \Gamma,\cutF \Uparrow L_1$
		\UnaryInf$\fCenter\Kcal_1: \Gamma,\cutF \Uparrow \nquest{i} P, L_1$
		\AxiomC{}
		\noLine
		\UnaryInfC{$\Pi_2$}
		\BinaryInfC{$\vdash \Kcal : \Gamma,\Delta\Uparrow \nquest{i} P, L_1,L_2$}
	\end{prooftree}
\end{minipage}
\begin{minipage}{0.1\textwidth}
	\begin{center}
		$\rightsquigarrow$
	\end{center}
\end{minipage}
\begin{minipage}{0.3\textwidth}
	\begin{prooftree}
		\AxiomC{}
		\noLine
		\UnaryInf$\fCenter\Kcal_1,P_i: \Gamma,\cutF \Uparrow L_1$
		\AxiomC{}
		\noLine
		\UnaryInfC{$\Pi_2$}
		\BinaryInf$\fCenter\Kcal,P_i: \Gamma,\Delta \Uparrow  L_1,L2$
		\UnaryInf$\fCenter \Kcal : \Gamma,\Delta\Uparrow\nquest{i} P, L_1,L_2$
	\end{prooftree}
\end{minipage}
\vspace{0.3cm}


	\vspace{0.3cm}	
\begin{minipage}{0.4\textwidth}
	\begin{prooftree}
		\AxiomC{}
		\noLine
		\UnaryInf$\fCenter\Kcal_1,P_i:  \Gamma,\cutF \Uparrow L_1$
		\UnaryInf$\fCenter\Kcal_1: \Gamma,\cutF \Uparrow \nquest{i} P, L_1$
		\AxiomC{}
		\noLine
		\UnaryInfC{$\Pi_2$}
		\BinaryInfC{$\vdash\Kcal : \Gamma,\Delta\Uparrow \nquest{i} P, L$}
	\end{prooftree}
\end{minipage}
\begin{minipage}{0.1\textwidth}
	\begin{center}
		$\rightsquigarrow$
	\end{center}
\end{minipage}
\begin{minipage}{0.3\textwidth}
	\begin{prooftree}
		\AxiomC{}
		\noLine
		\UnaryInf$\fCenter\Kcal_1,P_i:  \Gamma,\cutF \Uparrow L$
		\AxiomC{}
		\noLine
		\UnaryInfC{$\Pi_2$}
		\RightLabel{$\W$}
		\UnaryInfC{$\fCenter\Kcal_2,P_i: \Delta \Uparrow  \cutF^\bot,L_2$}
		\BinaryInf$\fCenter\Kcal,P_i: \Gamma,\Delta \Uparrow  L_1,L_2$
		\UnaryInf$\fCenter\Kcal : \Gamma,\Delta\Uparrow\nquest{i} P, L_1,L_2$
	\end{prooftree}
\end{minipage}
\vspace{0.3cm}
	
	\vspace{0.3cm}	
	\begin{minipage}{0.4\textwidth}
		\begin{prooftree}
			\AxiomC{}
			\noLine
			\UnaryInf$\fCenter\Kcal_1 : P, \Gamma,\cutF \Uparrow L_1$
			\UnaryInf$\fCenter\Kcal_1 : \Gamma,\cutF \Uparrow P, L_1$
			\AxiomC{}
			\noLine
			\UnaryInfC{$\Pi_2$}
			\BinaryInfC{$\vdash \Kcal :\Gamma,\Delta\Uparrow P, L_1,L_2$}
		\end{prooftree}
	\end{minipage}
	\begin{minipage}{0.1\textwidth}
		\begin{center}
			$\rightsquigarrow$
		\end{center}
	\end{minipage}
	\begin{minipage}{0.3\textwidth}
		\begin{prooftree}
			\AxiomC{}
			\noLine
			\UnaryInf$\fCenter \Kcal_1 : P, \Gamma,\cutF \Uparrow L_1$
			\AxiomC{}
			\noLine
			\UnaryInfC{$\Pi_2$}
			\BinaryInf$\fCenter \Kcal : P, \Gamma,\Delta \Uparrow  L_1,L_2$
			\UnaryInf$\fCenter \Kcal : \Gamma,\Delta\Uparrow P, L_1,L_2$
		\end{prooftree}
\end{minipage}
	\vspace{0.3cm}
	

	\vspace{0.3cm}
\begin{tcolorbox}
	\begin{minipage}{0.4\textwidth}
	\begin{prooftree}
			\AxiomC{}
			\noLine
			\UnaryInf$\fCenter\Kcal_1 : \Gamma,\cutF\Downarrow P$
			\UnaryInf$\fCenter\Kcal_1 : \Gamma,\cutF\Downarrow P\oplus Q$
			\UnaryInf$\fCenter\Kcal_1 : P\oplus Q,\Gamma,\cutF\Uparrow \cdot$
			\AxiomC{}
			\noLine
			\UnaryInfC{$\Pi_2$}
			\BinaryInfC{$\vdash \Kcal : P\oplus Q,\Gamma,\Delta\Uparrow L_2$}
	\end{prooftree}
	\end{minipage}
	\begin{minipage}{0.1\textwidth}
		\begin{center}
			$\rightsquigarrow$
		\end{center}
	\end{minipage}
	\begin{minipage}{0.3\textwidth}
	\begin{prooftree}
	\AxiomC{}
	\noLine
	\UnaryInf$\fCenter\Kcal_1 : \Gamma,\cutF\Downarrow P$
	\UnaryInf$\fCenter\Kcal_1 : P,\Gamma,\cutF\Uparrow \cdot$
	\AxiomC{}
	\noLine
	\UnaryInfC{$\Pi_2$}
	\BinaryInf$\fCenter \Kcal : P,\Gamma,\Delta\Uparrow  L_2$
	
	\UnaryInf$\fCenter \Kcal : \Gamma,\Delta\Uparrow L_2,P$
	\RightLabel{$\mathcal{L}\oplus$}
	\UnaryInf$\fCenter \Kcal : P\oplus Q,\Gamma,\Delta\Uparrow L_2$
	\end{prooftree}
	\end{minipage}
	\vspace{0.3cm}	

	\begin{minipage}{0.4\textwidth}
	\begin{prooftree}
		\AxiomC{}
		\noLine
		\UnaryInf$\fCenter\Kcal_1 : \Gamma,\cutF\Uparrow P$
		\UnaryInf$\fCenter\Kcal_1 : \Gamma,\cutF\Downarrow P$
		\UnaryInf$\fCenter\Kcal_1 : \Gamma,\cutF\Downarrow P\oplus Q$
		\UnaryInf$\fCenter\Kcal_1 : P\oplus Q,\Gamma,\cutF\Uparrow \cdot$
		\AxiomC{}
		\noLine
		\UnaryInfC{$\Pi_2$}
		\BinaryInfC{$\vdash \Kcal : P\oplus Q,\Gamma,\Delta\Uparrow L_2$}
	\end{prooftree}
\end{minipage}
\begin{minipage}{0.1\textwidth}
	\begin{center}
		$\rightsquigarrow$
	\end{center}
\end{minipage}
\begin{minipage}{0.3\textwidth}
	\begin{prooftree}
		\AxiomC{}
		\noLine
		\UnaryInf$\fCenter\Kcal_1 : \Gamma,\cutF\Uparrow P$
		\AxiomC{}
		\noLine
		\UnaryInfC{$\Pi_2$}
		\BinaryInf$\fCenter \Kcal : \Gamma,\Delta\Uparrow  P,L_2$
		
		\UnaryInf$\fCenter \Kcal : \Gamma,\Delta\Uparrow L_2,P$
		\RightLabel{$\mathcal{L}\oplus$}
		\UnaryInf$\fCenter \Kcal : P\oplus Q,\Gamma,\Delta\Uparrow L_2$
	\end{prooftree}
\end{minipage}
\vspace{0.3cm}	

If $P\oplus Q\in \Kcal_1$, we proceed similarly. For $\otimes$ and $\exists$ we use the same reasoning.


	\vspace{0.3cm}
\begin{minipage}{0.4\textwidth}
	\begin{prooftree}
		\AxiomC{}
		\noLine
		\UnaryInf$\fCenter \Kcal_1 : \Gamma\Downarrow \cutF$
		\UnaryInf$\fCenter \Kcal_1 : \Gamma,\cutF\Uparrow \cdot$
		\AxiomC{}
		\noLine
		\UnaryInf$\fCenter  \Kcal_2 : \Delta\Uparrow \cutF^{\bot},L_2$
		\BinaryInfC{$\vdash \Kcal : \Gamma,\Delta\Uparrow L_2$}
	\end{prooftree}
\end{minipage}
\begin{minipage}{0.1\textwidth}
	\begin{center}
		$\rightsquigarrow$
	\end{center}
\end{minipage}
\begin{minipage}{0.3\textwidth}
	\begin{prooftree}
		\AxiomC{}
		\noLine
		\UnaryInf$\fCenter \Kcal_2 : \Delta\Uparrow \cutF^{\bot},L_2$
		\AxiomC{}
		\noLine
		\UnaryInfC{$\vdash  \Kcal_1 : \Gamma \Downarrow \cutF$}
		\RightLabel{$\cut_2$}
		\BinaryInf$\fCenter \Kcal : \Gamma,\Delta\Uparrow \cdot$
	\end{prooftree}
\end{minipage}
\end{tcolorbox}

\vspace{0.3cm}	

\newpage
\section{Elimination of $\cut_2$}

{\footnotesize	
	\vspace{0.3cm}	
	
\begin{center}
	$	\infer[cut_2]{\unfoc{\Kcal}{\Gamma,\Delta}{L}}{
	\deduce{\unfoc{\Kcal_1}{\Gamma}{\cutF,L}}{}
	&
	\deduce{\foc{\Kcal_2}{\Delta}{\cutF^\perp}}{}
}$
\end{center}

	\vspace{0.3cm}	
}


{\footnotesize

	\vspace{0.3cm}
	\begin{minipage}{0.4\textwidth}
		\begin{prooftree}
			\AxiomC{}
			\noLine
			\UnaryInf$\fCenter\Kcal : \Gamma\Uparrow L$
			\UnaryInf$\fCenter\Kcal : \Gamma \Uparrow \bot, L$
			
			\AxiomC{}
			\UnaryInfC{$\fCenter\Kcal^u : \cdot \Downarrow 1$}
			\BinaryInfC{$\vdash\Kcal :\Gamma\Uparrow L$}
		\end{prooftree}
	\end{minipage}
	\begin{minipage}{0.2\textwidth}
		\begin{center}
			$\rightsquigarrow$
		\end{center}
	\end{minipage}
	\begin{minipage}{0.3\textwidth}
		\begin{prooftree}
			\AxiomC{}
			\noLine
			\UnaryInfC{$\vdash\Kcal:\Gamma\Uparrow L$}
		\end{prooftree}
	\end{minipage}
	\vspace{0.3cm}
	
		\vspace{0.3cm}
	\begin{minipage}{0.5\textwidth}
		\begin{prooftree}
			\AxiomC{[$\Sigma_1$]}
			\noLine
			\UnaryInf$\fCenter\Kcal_1:\Gamma\Uparrow P,Q,L$
			\UnaryInf$\fCenter\Kcal_1:\Gamma\Uparrow P \parr Q, L$
			\AxiomC{[$\Sigma_2$]}
			\noLine
			\UnaryInfC{$\vdash\Kcal'_2:\Delta_{P}\Downarrow P^{\bot}$}
			\AxiomC{[$\Sigma_3$]}
			\noLine
			\UnaryInfC{$\vdash\Kcal''_2:\Delta_{Q}\Downarrow Q^{\bot}$}
			\BinaryInfC{$\vdash\Kcal_2:\Delta\Downarrow P^{\bot}\otimes Q^{\bot}$}
			\BinaryInfC{$\vdash\Kcal:\Gamma,\Delta\Uparrow L $}
		\end{prooftree}
	\end{minipage}

	\begin{minipage}{0.5\textwidth}
		\begin{center}
			$\rightsquigarrow$
		\end{center}
	\end{minipage}
	\begin{minipage}{0.4\textwidth}
		\begin{prooftree}
			\AxiomC{}
			\noLine
			\UnaryInfC{[$\Sigma_1$]}
			\AxiomC{}
			\noLine
			\UnaryInfC{[$\Sigma_2$]}
			\BinaryInfC{$\vdash\Kcal',\Theta_2:\Gamma,\Delta_{P} \Uparrow Q,L$}
			\AxiomC{}
			\noLine
			\UnaryInfC{[$\Sigma_3$]}
			\BinaryInfC{$\vdash\Kcal:\Gamma,\Delta\Uparrow L$}
		\end{prooftree}
	\end{minipage}
	\vspace{0.3cm}
	
	\vspace{0.3cm}
\begin{minipage}{0.4\textwidth}
	\begin{prooftree}
		\AxiomC{}
		\noLine
		\UnaryInfC{$\fCenter\Kcal_1:\Gamma\Uparrow P,L$}
		\AxiomC{}
		\noLine
		\UnaryInfC{$\vdash\Kcal_1:\Gamma \Uparrow Q,L$}
		\BinaryInfC{$\fCenter\Kcal_1:\Gamma\Uparrow P \with Q,L$}
		\AxiomC{}
		\noLine
		\UnaryInf$\fCenter\Kcal_2:\Delta \Downarrow  P^{\bot}$
		\UnaryInf$\fCenter\Kcal_2:\Delta \Downarrow P^{\bot} \oplus Q^{\bot}$
		\BinaryInfC{$\vdash\Kcal:\Gamma,\Delta\Uparrow L$}
	\end{prooftree}
\end{minipage}

\begin{minipage}{0.5\textwidth}
	\begin{center}
		$\rightsquigarrow$
	\end{center}
\end{minipage}
\begin{minipage}{0.4\textwidth}
	\begin{prooftree}
		\AxiomC{}
		\noLine
		\UnaryInfC{$\vdash\Kcal_1:\Gamma \Uparrow P,L$}
		\AxiomC{}
		\noLine
		\UnaryInfC{$\vdash\Kcal_2:\Delta\Downarrow P^{\bot}$}
		\BinaryInfC{$\vdash\Kcal:\Gamma,\Delta\Uparrow L$}
	\end{prooftree}
\end{minipage}
\vspace{0.3cm}	


	\vspace{0.3cm}
\begin{minipage}{0.5\textwidth}
	\begin{prooftree}
		\AxiomC{}
		\noLine		
		\UnaryInf$\fCenter\Kcal_1:\Gamma\Uparrow P[c/x],L$
		\UnaryInf$\fCenter\Kcal_1:\Gamma\Uparrow \forall x.P,L$
		\AxiomC{}
		\noLine
		\UnaryInf$\fCenter\Kcal_2:\Delta \Downarrow  P^{\bot}[t/x]$
		\UnaryInf$\fCenter\Kcal_2:\Delta \Downarrow \exists  x.P^{\bot}$
		\BinaryInfC{$\vdash\Kcal:\Gamma,\Delta\Uparrow L$}
	\end{prooftree}
\end{minipage}

\begin{minipage}{0.5\textwidth}
	\begin{center}
		$\rightsquigarrow$
	\end{center}
\end{minipage}
\begin{minipage}{0.4\textwidth}
	\begin{prooftree}
		\AxiomC{}
		\noLine
		\UnaryInfC{$\fCenter\Kcal_1:\Gamma\Uparrow P[c/x],L$}
		\AxiomC{}
		\noLine		
		\UnaryInfC{$\fCenter\Kcal_2:\Delta \Downarrow  P^{\bot}[t/x]$}	
		\BinaryInfC{$\vdash\Kcal:\Gamma,\Delta\Uparrow L$}
	\end{prooftree}
\end{minipage}
\vspace{0.3cm}

\vspace{0.3cm}
\begin{tcolorbox}
\begin{minipage}{0.5\textwidth}
	\vspace{0.3cm}	
	\begin{prooftree}
		\AxiomC{}
		\noLine
		\UnaryInf$\fCenter\Kcal_1,H_i : \Gamma\Uparrow L$
		\UnaryInf$\fCenter\Kcal_1 : \Gamma\Uparrow \nquest{i} H, L$
		\AxiomC{}
		\noLine
		\UnaryInf$\fCenter\Kcal_2 : \cdot\Downarrow \nbang{i} H$
		\BinaryInfC{$\vdash\Kcal : \Gamma\Uparrow L $}
	\end{prooftree}
	\vspace{0.3cm}	
\end{minipage}
\begin{minipage}{0.1\textwidth}
	\begin{center}
		$\rightsquigarrow$
	\end{center}
\end{minipage}
\begin{minipage}{0.2\textwidth}
	\begin{prooftree}
		\AxiomC{}
		\noLine
		\UnaryInf$\fCenter\Kcal_1,H_i : \Gamma\Uparrow L$
		\AxiomC{}
		\noLine
		\UnaryInf$\fCenter\Kcal_2 : \cdot\Downarrow \nbang{i} H$
		\RightLabel{$\cut_3$}
		\BinaryInfC{$\vdash\Kcal : \Gamma\Uparrow L $}
	\end{prooftree}
\end{minipage}
\vspace{0.3cm}

\vspace{0.3cm}	
\begin{minipage}{0.4\textwidth}
\begin{prooftree}
	\AxiomC{}
	\noLine
		\UnaryInf$\fCenter \Kcal_1 : H,\Gamma\Uparrow L$
	\UnaryInf$\fCenter \Kcal_1:\Gamma\Uparrow H, L$
	\AxiomC{}
	\noLine
	\UnaryInfC{$\fCenter \Kcal_2:\Delta\Uparrow H^{\bot}$}
	\UnaryInfC{$\fCenter \Kcal_2:\Delta\Downarrow H^{\bot}$}
	\BinaryInfC{$\vdash \Kcal:\Gamma,\Delta\Uparrow L $}
\end{prooftree}
\vspace{0.3cm}	
	\end{minipage}
	\begin{minipage}{0.15\textwidth}
		\begin{center}
			$\rightsquigarrow$
		\end{center}
	\end{minipage}
	\begin{minipage}{0.15\textwidth}
\begin{prooftree}
	\AxiomC{}
	\noLine
	\UnaryInf$\fCenter \Kcal_1:H,\Gamma\Uparrow L$
	\AxiomC{}
	\noLine
	\UnaryInfC{$\fCenter \Kcal_2:\Delta\Downarrow H^{\bot}$}
	\RightLabel{$\cut_1$}
	\BinaryInfC{$\vdash \Kcal:\Gamma,\Delta\Uparrow L $}
\end{prooftree}
	\end{minipage}


\vspace{0.6cm}	
\begin{minipage}{0.4\textwidth}
	\begin{prooftree}
		\AxiomC{}
		\noLine
		\UnaryInf$\fCenter \Kcal_1 :A^{+},\Gamma\Uparrow L$
		\UnaryInf$\fCenter \Kcal_1:\Gamma\Uparrow A^{+}, L$
		\AxiomC{}
		\UnaryInfC{$\fCenter \Kcal_2:A^{+}\Downarrow A^{-}$}
		\BinaryInfC{$\vdash \Kcal_1:\Gamma,A^{+}\Uparrow L $}
	\end{prooftree}
	\vspace{0.3cm}	
\end{minipage}

\vspace{0.6cm}	
\begin{minipage}{0.4\textwidth}
	\begin{prooftree}
		\AxiomC{}
		\noLine
		\UnaryInf$\fCenter \Kcal_1 : A^+,\Gamma\Uparrow L$
		\UnaryInf$\fCenter \Kcal_1:\Gamma\Uparrow A^+, L$
		\AxiomC{}
		\UnaryInfC{$\fCenter \Kcal_2,A^+_i:\cdot\Downarrow A^{-}$}
		\BinaryInfC{$\vdash \Kcal,A^+_i:\Gamma\Uparrow L $}
	\end{prooftree}
	\vspace{0.3cm}	
\end{minipage}
\begin{minipage}{0.15\textwidth}
	\begin{center}
		$\rightsquigarrow$
	\end{center}
\end{minipage}
\begin{minipage}{0.15\textwidth}
	\begin{prooftree}
		\AxiomC{}
		\noLine
		\UnaryInfC{$\vdash \Kcal_1:\Gamma,A^+\Uparrow L $}
		\RightLabel{$\A_l$}
		\UnaryInfC{$\vdash \Kcal,A^+_i:\Gamma\Uparrow L $}
	\end{prooftree}
\end{minipage}

\vspace{0.6cm}	
\begin{minipage}{0.4\textwidth}
	\begin{prooftree}
		\AxiomC{}
		\noLine
		\UnaryInf$\fCenter \Kcal_1,A^+_i : A^+,\Gamma\Uparrow L$
		\UnaryInf$\fCenter \Kcal_1,A^+_i:\Gamma\Uparrow A^+, L$
		\AxiomC{}
		\UnaryInfC{$\fCenter \Kcal_2,A^+_i:\cdot\Downarrow A^{-}$}
		\BinaryInfC{$\vdash \Kcal,A^+_i:\Gamma\Uparrow L $}
	\end{prooftree}
	\vspace{0.3cm}	
\end{minipage}
\begin{minipage}{0.15\textwidth}
	\begin{center}
		$\rightsquigarrow$
	\end{center}
\end{minipage}
\begin{minipage}{0.15\textwidth}
	\begin{prooftree}
		\AxiomC{}
		\noLine
		\UnaryInfC{$\vdash \Kcal_1,A^+_i:\Gamma,A^+\Uparrow L $}
		\RightLabel{$\A_c$}
		\UnaryInfC{$\vdash \Kcal,A^+_i:\Gamma\Uparrow L $}
	\end{prooftree}
\end{minipage}

\end{tcolorbox}
	\vspace{0.3cm}}
\newpage
\section{Elimination of $\cut_3$}
	
{\footnotesize
	\vspace{0.2cm}
\begin{center}
	$	\infer[\cut_3]{\unfoc{\Kcal}{\Gamma}{L}}{
	\deduce{\unfoc{\Kcal_1,\cutF_i}{\Gamma}{L}}{}
	&
	\deduce{\foc{\Kcal_2}{\cdot}{\nbang{i}\cutF^\perp}}{}
}$
\end{center}
	\vspace{0.2cm}
}
	
	{\footnotesize	
	\vspace{0.3cm}	
	\begin{minipage}{0.4\textwidth}
		\begin{prooftree}
			\AxiomC{}
			\noLine
			\UnaryInf$\fCenter\Kcal_1,\cutF_i : \Gamma \Uparrow  L$
			\UnaryInf$\fCenter\Kcal_1,\cutF_i : \Gamma \Uparrow  \bot, L$
			\AxiomC{}
			\noLine
			\UnaryInfC{$\Pi_2$}
			\BinaryInfC{$\vdash\Kcal: \Gamma\Uparrow \bot, L$}
		\end{prooftree}
	\end{minipage}
	\begin{minipage}{0.1\textwidth}
		\begin{center}
			$\rightsquigarrow$
		\end{center}
	\end{minipage}
	\begin{minipage}{0.3\textwidth}
		\begin{prooftree}
			\AxiomC{}
			\noLine
			\UnaryInfC{$\vdash\Kcal_1,\cutF_i : \Gamma \Uparrow  L$}
			\AxiomC{}
			\noLine
			\UnaryInfC{$\Pi_2$}
			\BinaryInf$\fCenter\Kcal: \Gamma\Uparrow L$
			\UnaryInf$\fCenter\Kcal: \Gamma\Uparrow \bot, L$
		\end{prooftree}
	\end{minipage}
	\vspace{0.3cm}
	
	\vspace{0.3cm}
	\begin{minipage}{0.4\textwidth}
		\begin{prooftree}
			\AxiomC{}
			\UnaryInfC{$\vdash\Kcal_1,\cutF_i : \Gamma \Uparrow  \top, L$}
			\AxiomC{}
			\noLine
			\UnaryInfC{$\Pi_2$}
			\BinaryInfC{$\vdash\Kcal: \Gamma\Uparrow \top, L$}
		\end{prooftree}
	\end{minipage}
	\begin{minipage}{0.1\textwidth}
		\begin{center}
			$\rightsquigarrow$
		\end{center}
	\end{minipage}
	\begin{minipage}{0.3\textwidth}
		\begin{prooftree}
			\AxiomC{}
			\UnaryInfC{$\vdash\Kcal: \Gamma\Uparrow \top, L$}
		\end{prooftree}
	\end{minipage}
	\vspace{0.3cm}
	
	\vspace{0.3cm}
	\begin{minipage}{0.4\textwidth}
		\begin{prooftree}
			\AxiomC{}
			\noLine
			\UnaryInf$\fCenter\Kcal_1,\cutF_i : \Gamma \Uparrow  P, Q, L$
			\UnaryInf$\fCenter\Kcal_1,\cutF_i : \Gamma \Uparrow  P\parr Q, L$
			\AxiomC{}
			\noLine
			\UnaryInfC{$\Pi_2$}
			\BinaryInfC{$\vdash\Kcal:\Gamma\Uparrow P\parr Q, L$}
		\end{prooftree}
	\end{minipage}
	\begin{minipage}{0.1\textwidth}
		\begin{center}
			$\rightsquigarrow$
		\end{center}
	\end{minipage}
	\begin{minipage}{0.3\textwidth}
		\begin{prooftree}
			\AxiomC{}
			\noLine
			\UnaryInfC{$\vdash\Kcal_1,\cutF_i : \Gamma \Uparrow  P, Q, L$}
			\AxiomC{}
			\noLine
			\UnaryInfC{$\Pi_2$}
			\BinaryInf$\fCenter\Kcal: \Gamma\Uparrow P, Q, L$
			\UnaryInf$\fCenter\Kcal: \Gamma\Uparrow P\parr Q, L  $
		\end{prooftree}
	\end{minipage}
	\vspace{0.3cm}
	
	\vspace{0.3cm}	
	\begin{minipage}{0.4\textwidth}
		\begin{prooftree}
			\AxiomC{$[\Sigma_1]$}
			\noLine
			\UnaryInfC{$\vdash\Kcal_1,\cutF_i : \Gamma \Uparrow  P, L$}
			\AxiomC{$[\Sigma_2]$}
			\noLine
			\UnaryInfC{$\vdash\Kcal_1,\cutF_i : \Gamma \Uparrow  Q, L$}
			\BinaryInfC{$\vdash\Kcal_1,\cutF_i : \Gamma \Uparrow  P\with Q, L$}
			\AxiomC{}
			\noLine
			\UnaryInfC{$\Pi_2$}
			\BinaryInfC{$\vdash\Kcal: \Gamma\Uparrow P\with Q, L$}
		\end{prooftree}
	\end{minipage}

	\begin{minipage}{0.5\textwidth}
		\begin{center}
			$\rightsquigarrow$
		\end{center}
	\end{minipage}
	\begin{minipage}{0.4\textwidth}
		\begin{prooftree}
			\AxiomC{}
			\noLine
			\UnaryInfC{$\Sigma_1$}
			\AxiomC{}
			\noLine
			\UnaryInfC{$\Pi_2$}
			\BinaryInfC{$\vdash\Kcal : \Gamma \Uparrow  P, L$}
			\AxiomC{}
			\noLine
			\UnaryInfC{$\Sigma_2$}
			\AxiomC{}
			\noLine
			\UnaryInfC{$\Pi_2$}
			\BinaryInfC{$\vdash\Kcal: \Gamma \Uparrow  Q, L$}
			\BinaryInfC{$\vdash\Kcal: \Gamma\Uparrow P\with Q, L$}
		\end{prooftree}
	\end{minipage} 
	\vspace{0.3cm}

	\vspace{0.3cm}
\begin{minipage}{0.4\textwidth}
	\begin{prooftree}
		\AxiomC{}
		\noLine
		\UnaryInf$\fCenter\Kcal_1,\cutF_i,P_a : \Gamma \Uparrow  L$
		\UnaryInf$\fCenter\Kcal_1,\cutF_i : \Gamma \Uparrow  \nquest{a} P,L$
		\AxiomC{}
		\noLine
		\UnaryInfC{$\Pi_2$}
		\BinaryInfC{$\vdash\Kcal: \Gamma\Uparrow \nquest{a} P,L$}
	\end{prooftree}
\end{minipage}
\begin{minipage}{0.1\textwidth}
	\begin{center}
		$\rightsquigarrow$
	\end{center}
\end{minipage}
\begin{minipage}{0.3\textwidth}
	\begin{prooftree}
		\AxiomC{}
		\noLine
		\UnaryInfC{$\vdash\Kcal_1,P_a,\cutF_i : \Gamma \Uparrow L$}
		\AxiomC{}
		\noLine
		\UnaryInfC{$\Pi_2$}
		\BinaryInf$\fCenter\Kcal, P_a : \Gamma\Uparrow  L$
		\UnaryInf$\fCenter\Kcal: \Gamma\Uparrow \nquest{a} P, L$
	\end{prooftree}
\end{minipage} 
\vspace{0.3cm}

	
	\vspace{0.3cm}
	\begin{minipage}{0.4\textwidth}
		\begin{prooftree}
			\AxiomC{}
			\noLine
			\UnaryInf$\fCenter\Kcal_1,\cutF_i,P_a : \Gamma \Uparrow  L$
			\UnaryInf$\fCenter\Kcal_1,\cutF_i : \Gamma \Uparrow \nquest{a} P,L$
			\AxiomC{}
			\noLine
			\UnaryInfC{$\Pi_2$}
			\BinaryInfC{$\vdash\Kcal: \Gamma\Uparrow \nquest{a} P,L$}
		\end{prooftree}
	\end{minipage}
	\begin{minipage}{0.1\textwidth}
		\begin{center}
			$\rightsquigarrow$
		\end{center}
	\end{minipage}
	\begin{minipage}{0.3\textwidth}
		\begin{prooftree}
			\AxiomC{}
			\noLine
			\UnaryInfC{$\vdash\Kcal_1,P_a,\cutF_i : \Gamma \Uparrow L$}
			\AxiomC{}
			\noLine
			\UnaryInfC{$\Pi_2$}
			\RightLabel{$\W$}
			\UnaryInfC{$\fCenter\Kcal_2,P_a: \cdot \Downarrow  \nbang{i} H^{\bot}$}
			\BinaryInf$\fCenter\Kcal, P_a : \Gamma\Uparrow  L$
			\UnaryInf$\fCenter\Kcal: \Gamma\Uparrow \nquest{a} P, L$
		\end{prooftree}
	\end{minipage} 
	\vspace{0.3cm}
	
	\vspace{0.3cm}
	\begin{minipage}{0.4\textwidth}
		\begin{prooftree}
			\AxiomC{}
			\noLine
			\UnaryInf$\fCenter\Kcal_1, \cutF_i : \Gamma \Uparrow  P[c/x], L$
			\UnaryInf$\fCenter\Kcal_1,\cutF_i : \Gamma \Uparrow  \forall xP,L$
			\AxiomC{}
			\noLine
			\UnaryInfC{$\Pi_2$}
			\BinaryInfC{$\vdash\Kcal: \Gamma\Uparrow \forall xP,L$}
		\end{prooftree}
	\end{minipage}
	\begin{minipage}{0.1\textwidth}
		\begin{center}
			$\rightsquigarrow$
		\end{center}
	\end{minipage}
	\begin{minipage}{0.3\textwidth}
		\begin{prooftree}
			\AxiomC{}
			\noLine
			\UnaryInfC{$\vdash\Kcal_1,\cutF_i : \Gamma \Uparrow P[c/x], L$}
			\AxiomC{}
			\noLine
			\UnaryInfC{$\Pi_2$}
			\BinaryInf$\fCenter\Kcal:\Gamma\Uparrow  P[c/x], L$
			\UnaryInf$\fCenter\Kcal: \Gamma\Uparrow \forall xP, L$
		\end{prooftree}
	\end{minipage} 
	\vspace{0.3cm}
	
	\vspace{0.3cm}	
	\begin{minipage}{0.4\textwidth}
		\begin{prooftree}
			\AxiomC{}
			\noLine
			\UnaryInf$\fCenter\Kcal_1,\cutF_i : P, \Gamma \Uparrow  L$
			\UnaryInf$\fCenter\Kcal_1,\cutF_i : \Gamma \Uparrow  P, L$
			\AxiomC{}
			\noLine
			\UnaryInfC{$\Pi_2$}
			\BinaryInfC{$\vdash\Kcal: \Gamma\Uparrow P, L$}
		\end{prooftree}
	\end{minipage}
	\begin{minipage}{0.1\textwidth}
		\begin{center}
			$\rightsquigarrow$
		\end{center}
	\end{minipage}
	\begin{minipage}{0.3\textwidth}
		\begin{prooftree}
			\AxiomC{}
			\noLine
			\UnaryInfC{$\vdash\Kcal_1,\cutF_i : P, \Gamma \Uparrow L$}
			\AxiomC{}
			\noLine
			\UnaryInfC{$\Pi_2$}
			\BinaryInf$\fCenter\Kcal: P, \Gamma \Uparrow  L$
			\UnaryInf$\fCenter\Kcal: \Gamma\Uparrow P, L$
		\end{prooftree}
	\end{minipage} 
	\vspace{0.3cm}
	
	
	\vspace{0.3cm}		
\begin{tcolorbox}
	
	\begin{minipage}{0.4\textwidth}
		
		\begin{prooftree}
			\AxiomC{}
			\noLine
			\UnaryInf$\fCenter\Kcal_1,\cutF_i : \Gamma\Downarrow Q$
			\UnaryInf$\fCenter\Kcal_1, \cutF_i : \Gamma\Uparrow \cdot$
			\AxiomC{}
			\noLine
			\UnaryInf$\fCenter\Kcal_2, \cutF_i : \cdot\Uparrow \nbang{i} \cutF^\perp$
			\BinaryInfC{$\vdash\Kcal: \Gamma\Uparrow \cdot$}
		\end{prooftree}
	\end{minipage}
	\begin{minipage}{0.1\textwidth}
		\begin{center}
			$\rightsquigarrow$
		\end{center}
	\end{minipage}
	\begin{minipage}{0.3\textwidth}
		\begin{prooftree}
			\AxiomC{}
			\noLine
			\UnaryInfC{$\vdash\Kcal_1, \cutF_i : \Gamma \Downarrow Q$}
			\AxiomC{}
			\noLine
			\UnaryInf$\fCenter\Kcal_2, \cutF_i : \cdot\Uparrow \nbang{i} \cutF^\perp$
			\RightLabel{$\cut_4$}
			\BinaryInf$\fCenter\Kcal: \Gamma\Uparrow Q$
			\RightLabel{$\mathcal{L}A$}
			\UnaryInf$\fCenter\Kcal: \Gamma\Uparrow \cdot $
			
		\end{prooftree}
	\end{minipage}
	\vspace{0.3cm}

	\vspace{0.3cm}
	\begin{minipage}{0.4\textwidth}
	\begin{prooftree}
		\AxiomC{}
		\noLine
		\UnaryInf$\fCenter\Kcal'_1,\cutF_i : \Gamma\Downarrow Q$
		\UnaryInf$\fCenter \Kcal_1, \cutF_i : \Gamma\Uparrow \cdot$
		\AxiomC{}
		\noLine
		\UnaryInf$\fCenter\Kcal_2, \cutF_i : \cdot\Uparrow \nbang{i} \cutF^\perp$
		\BinaryInfC{$\vdash \Kcal: \Gamma\Uparrow \cdot$}
	\end{prooftree}
\end{minipage}
\begin{minipage}{0.1\textwidth}
	\begin{center}
		$\rightsquigarrow$
	\end{center}
\end{minipage}
\begin{minipage}{0.3\textwidth}
	\begin{prooftree}
		\AxiomC{}
		\noLine
		\UnaryInfC{$\vdash\Kcal'_1, \cutF_i : \Gamma \Downarrow Q$}
		\AxiomC{}
		\noLine
		\UnaryInf$\fCenter\Kcal_2, \cutF_i : \cdot\Uparrow \nbang{i} \cutF^\perp$
		\RightLabel{$\cut_4$}
		\BinaryInf$\fCenter\Kcal': \Gamma\Uparrow Q$
		\RightLabel{$\mathcal{L}A*$}
		\UnaryInf$\fCenter \Kcal: \Gamma\Uparrow \cdot $
	\end{prooftree}
\end{minipage}
\vspace{0.3cm}
	
	\vspace{0.3cm}
	\begin{minipage}{0.4\textwidth}
		\begin{prooftree}
			\AxiomC{}
			\noLine
			\UnaryInf$\fCenter\Kcal_1, \cutF_i : \Gamma\Downarrow P$
			\UnaryInf$\fCenter\Kcal_1, \cutF_i : P,\Gamma\Uparrow \cdot$
			\AxiomC{}
			\noLine
			\UnaryInf$\fCenter\Kcal_2, \cutF_i : \cdot\Uparrow \nbang{i} \cutF^\perp$
			\BinaryInfC{$\vdash\Kcal: P,\Gamma\Uparrow \cdot$}
		\end{prooftree}
	\end{minipage}
	\begin{minipage}{0.1\textwidth}
		\begin{center}
			$\rightsquigarrow$
		\end{center}
	\end{minipage}
	\begin{minipage}{0.3\textwidth}
		\begin{prooftree}
			\AxiomC{}
			\noLine
			\UnaryInfC{$\vdash\Kcal_1, \cutF_i : \Gamma \Downarrow P$}
			\AxiomC{}
			\noLine
			\UnaryInf$\fCenter\Kcal_2, \cutF_i : \cdot\Uparrow \nbang{i} \cutF^\perp$
			\RightLabel{$\cut_4$}
				\BinaryInf$\fCenter\Kcal: \Gamma\Uparrow P$
			\UnaryInf$\fCenter\Kcal: P,\Gamma\Uparrow \cdot $
		\end{prooftree}
	\end{minipage}
	\vspace{0.4cm}
	
\end{tcolorbox}
\vspace{0.6cm}

\begin{tcolorbox}	
	\vspace{0.3cm}
	\begin{minipage}{0.4\textwidth}
		\begin{prooftree}
			\AxiomC{}
			\noLine
			\UnaryInf$\fCenter\Kcal_1,\cutF_c : \Gamma\Downarrow \cutF$
			\UnaryInf$\fCenter\Kcal_1,\cutF_c: \Gamma\Uparrow \cdot$
			\AxiomC{}
			\noLine
			\UnaryInf$\fCenter\Kcal_2: \cdot\Uparrow \cutF^{\bot}$
			\UnaryInf$\fCenter\Kcal_2 : \cdot\Downarrow\nbang{c} \cutF^{\bot}$
			\BinaryInfC{$\fCenter\Kcal: \Gamma\Uparrow \cdot$}
		\end{prooftree}
	\end{minipage}
\vspace{0.4cm}

We should analyze if $\cutF$ is (1) a negative ou (2) a positive formula. Note that $\Kcal_2$ is unbounded.

\vspace{0.4cm}

	\begin{minipage}{0.1\textwidth}
	\begin{center}
		$1 \rightsquigarrow$
	\end{center}
\end{minipage}
	\begin{minipage}{0.3\textwidth}
		\begin{prooftree}
			\AxiomC{}
			\noLine
			\UnaryInfC{$\vdash\Kcal_2 : \cdot \Uparrow \cutF^{\bot}$}
		
				\AxiomC{}
			\noLine
			\UnaryInfC{$\vdash\Kcal_1, \cutF_{c} : \Gamma \Downarrow \cutF$}
			
				\AxiomC{}
			\noLine
			\UnaryInfC{$\vdash\Kcal_2 : \cdot \Downarrow\nbang{c} \cutF^{\bot}$}
			\RightLabel{$\cut_2$}
			\BinaryInfC{$\fCenter\Kcal_1: \Gamma\Uparrow \cutF$}
	         \UnaryInfC{$\fCenter\Kcal_1: \Gamma\Downarrow \cutF$}
			
			\RightLabel{$\cut_2$}
			\BinaryInfC{$\vdash\Kcal_1:\Gamma\Uparrow \cdot$}
		
		\end{prooftree}
	\end{minipage}
	\vspace{0.6cm}

\begin{minipage}{0.1\textwidth}
	\begin{center}
		$2 \rightsquigarrow$
	\end{center}
\end{minipage}
\begin{minipage}{0.3\textwidth}
	\begin{prooftree}
				\AxiomC{}
\noLine
\UnaryInfC{$\vdash\Kcal_1, \cutF_{c} : \Gamma \Downarrow \cutF$}

\AxiomC{}
\noLine
\UnaryInfC{$\vdash\Kcal_2: \cdot \Downarrow\nbang{c} \cutF^{\bot}$}
\BinaryInfC{$\fCenter\Kcal_1: \Gamma\Uparrow \cutF$}
		
		\AxiomC{}
		\noLine
		\UnaryInfC{$\vdash\Kcal_2 : \cdot \Uparrow \cutF^{\bot}$}
		\UnaryInfC{$\vdash\Kcal_2 : \cdot \Downarrow \cutF^{\bot}$}
		
		\RightLabel{$\cut_2$}
		\BinaryInfC{$\vdash\Kcal_1:\Gamma\Uparrow \cdot$}
		
	\end{prooftree}
\end{minipage}
\end{tcolorbox}
\vspace{0.6cm}

\begin{tcolorbox}	
	\vspace{0.3cm}
	\begin{minipage}{0.4\textwidth}
		\begin{prooftree}
			\AxiomC{}
			\noLine
			\UnaryInf$\fCenter\Kcal_1 : \Gamma\Downarrow \cutF$
			\UnaryInf$\fCenter\Kcal_1,\cutF_{i}: \Gamma\Uparrow \cdot$
			\AxiomC{}
			\noLine
			\UnaryInf$\fCenter\Upsilon,\Sigma^u_{c}: \cdot\Uparrow \cutF^{\bot},\mathcal{F}(\Sigma^l)$
			\doubleLine
			\UnaryInf$\fCenter\Kcal_2 : \cdot\Uparrow \cdot\lns{i}\fCenter\cdot : \cdot\Uparrow \cutF^{\bot}$
			
			\UnaryInf$\fCenter\Kcal_2 : \cdot\Downarrow\nbang{i} \cutF^{\bot}$
			\BinaryInfC{$\fCenter\Kcal: \Gamma\Uparrow \cdot$}
		\end{prooftree}
	\end{minipage}
	\vspace{0.6cm}
	
	\begin{multicols}{2}[]
		{
			\begin{itemize}
				\item 	$\Kcal_2\equiv\Upsilon^4,\Sigma^{\bcancel{4}},\Xi^{U}$ where $\Upsilon_{_i\preceq}$ and $\Sigma_{_i\preceq}$
				\item $T \in \mathcal{U}(i)$, therefore $\Upsilon^T$ and $\Sigma^T$.
			\end{itemize}
		}
	\end{multicols}
\vspace{0.4cm}


	\begin{minipage}{0.1\textwidth}
		\begin{center}
		$\rightsquigarrow$
		\end{center}
	\end{minipage}
	\begin{minipage}{0.3\textwidth}
		\begin{prooftree}
			\AxiomC{}
			\noLine
			\UnaryInfC{$\vdash\Upsilon,  \Sigma^u_{c} : \cdot \Uparrow \cutF^{\bot},\mathcal{F}(\Sigma^l)$}
			\RightLabel{$\W$}
			\UnaryInfC{$\vdash\Upsilon,\Sigma^u,\Xi,\Sigma^u_{c} : \cdot \Uparrow \cutF^{\bot},\mathcal{F}(\Sigma^l)$}
			
			\AxiomC{}
			\noLine
			\UnaryInfC{$\fCenter\Kcal_1: \Gamma\Downarrow \cutF$}
				\RightLabel{$\W$}
			\UnaryInfC{$\fCenter\Kcal^l_1,\Upsilon^u,\Sigma^u,\Xi,\Sigma^u_{c}: \Gamma\Downarrow \cutF$}
			
			\RightLabel{$\cut_2$}
			\BinaryInfC{$\vdash\Kcal^l_1,\Upsilon,\Sigma^u,\Xi,\Sigma^u_{c}:\Gamma\Uparrow \mathcal{F}(\Sigma^l)$}
					\doubleLine
			\RightLabel{$\A_l$, $\C_c$}
			\UnaryInfC{$\vdash\Kcal:\Gamma\Uparrow \cdot$}
			
		\end{prooftree}
	\end{minipage}
\vspace{0.4cm}

Note that $\Kcal_1\equiv\Kcal^l_1, \Upsilon^U, \Sigma^U, \Xi$
\end{tcolorbox}
\vspace{0.6cm}

\begin{tcolorbox}	
	\vspace{0.3cm}
	\begin{minipage}{0.4\textwidth}
		\begin{prooftree}
			\AxiomC{}
			\noLine
			\UnaryInf$\fCenter\Kcal_1,\cutF_{i} : \Gamma\Downarrow \cutF$
			\UnaryInf$\fCenter\Kcal_1,\cutF_{i}: \Gamma\Uparrow \cdot$
			\AxiomC{}
			\noLine
			\UnaryInf$\fCenter\Upsilon,\Sigma_{c}: \cdot\Uparrow \cutF^{\bot}$
			\doubleLine
			\UnaryInf$\fCenter\Kcal_2 : \cdot\Uparrow \cdot\lns{i}\fCenter\cdot : \cdot\Uparrow \cutF^{\bot}$
			
			\UnaryInf$\fCenter\Kcal_2 : \cdot\Downarrow\nbang{i} \cutF^{\bot}$
			\BinaryInfC{$\fCenter\Kcal: \Gamma\Uparrow \cdot$}
		\end{prooftree}
	\end{minipage}
	\vspace{0.6cm}
	
	\begin{multicols}{2}[]
		{
			\begin{itemize}
				\item 	$\Kcal_2\equiv\Upsilon^{U4},\Sigma^{U\bcancel{4}},\Xi^{U}$ where $\Upsilon_{_i\preceq}$ and $\Sigma_{_i\preceq}$
				\item $T \in \mathcal{U}(i)$, therefore $\Upsilon^T$ and $\Sigma^T$.
			\end{itemize}
		}
	\end{multicols}
	\vspace{0.4cm}
	
We should analyze if $\cutF$ is (1) a negative ou (2) a positive formula. Note that $\Kcal_2$ is unbounded.

\vspace{0.4cm}

\begin{minipage}{0.1\textwidth}
	\begin{center}
		$1 \rightsquigarrow$
	\end{center}
\end{minipage}
\begin{minipage}{0.3\textwidth}
	\begin{prooftree}
		\AxiomC{}
		\noLine
		\UnaryInfC{$\vdash\Upsilon, \Sigma_c : \cdot \Uparrow \cutF^{\bot}$}
		\RightLabel{$\W$}
		\UnaryInfC{$\vdash\Kcal_2, \Sigma_c : \cdot \Uparrow \cutF^{\bot}$}
		
		\AxiomC{}
		\noLine
		\UnaryInfC{$\vdash\Kcal_1, \cutF_i : \Gamma \Downarrow \cutF$}
		
		\AxiomC{}
		\noLine
		\UnaryInfC{$\vdash\Kcal_2 : \cdot \Downarrow\nbang{i} \cutF^{\bot}$}
		\BinaryInfC{$\fCenter\Kcal: \Gamma\Uparrow \cutF$}
		\RightLabel{$\W$}
		\UnaryInfC{$\fCenter\Kcal, \Sigma_c: \Gamma\Downarrow \cutF$}
		
		\RightLabel{$\cut_2$}
		\BinaryInfC{$\vdash\Kcal, \Sigma_c:\Gamma\Uparrow \cdot$}
			\doubleLine
\RightLabel{$\C_c$}
\UnaryInfC{$\vdash\Kcal:\Gamma\Uparrow \cdot$}
		
	\end{prooftree}
\end{minipage}
\vspace{0.6cm}

\begin{minipage}{0.1\textwidth}
	\begin{center}
		$2 \rightsquigarrow$
	\end{center}
\end{minipage}
\begin{minipage}{0.3\textwidth}
	\begin{prooftree}
		\AxiomC{}
		\noLine
		\UnaryInfC{$\vdash\Kcal_1, \cutF_{i} : \Gamma \Downarrow \cutF$}
		
		\AxiomC{}
		\noLine
		\UnaryInfC{$\vdash\Kcal_2: \cdot \Downarrow\nbang{i} \cutF^{\bot}$}
		\BinaryInfC{$\fCenter\Kcal: \Gamma\Uparrow \cutF$}
		\RightLabel{$\W$}
		\UnaryInfC{$\fCenter\Kcal,\Sigma_c: \Gamma\Uparrow \cutF$}
		
		\AxiomC{}
		\noLine
		\UnaryInfC{$\vdash\Upsilon,\Sigma_c : \cdot \Uparrow \cutF^{\bot}$}
		\RightLabel{$\W$}
		\UnaryInfC{$\vdash\Kcal_2,\Sigma_c  : \cdot \Downarrow \cutF^{\bot}$}
		
		\RightLabel{$\cut_2$}
		\BinaryInfC{$\vdash\Kcal, \Sigma_c:\Gamma\Uparrow \cdot$}
\doubleLine
\RightLabel{$\C_c$}
\UnaryInfC{$\vdash\Kcal:\Gamma\Uparrow \cdot$}
		
	\end{prooftree}
\end{minipage}
	\vspace{0.4cm}
	
\end{tcolorbox}
\vspace{0.6cm}
	
}
\newpage
\section{Elimination of $\cut_4$}
{\footnotesize
	\vspace{0.2cm}

\begin{center}
	$\infer[cut_4]{\unfoc{\Kcal}{\Gamma}{F}}{
	\deduce{\foc{\Kcal_1,\cutF_i}{\Gamma}{F}}{}
	&
	\deduce{\foc{\Kcal_2}{\cdot}{\nbang{i} \cutF^\perp}}{}
}$
\end{center}
}

\begin{tcolorbox}[colback=red!5!white]
	Note that $\vdash\Theta : \Gamma\Downarrow P$ implies $\vdash\Theta : \Gamma\Uparrow P$.
	
%	\begin{multicols}{2}[]
%		{
%			\begin{center}
%				1. $P$ is a positive formula.
%			\end{center}
%			
%			\begin{prooftree}
%				\AxiomC{}
%				\noLine
%				\UnaryInfC{$\vdash\Theta : \Gamma\Downarrow P$}
%				\UnaryInfC{$\vdash\Theta : \Gamma,P\Uparrow \cdot$}
%				\UnaryInfC{$\vdash\Theta : \Gamma\Uparrow P$}
%			\end{prooftree}
%			
%			\begin{center}
%				2. $P$ is a negative formula.
%			\end{center}
%			
%			\begin{prooftree}
%				\AxiomC{}
%				\noLine
%				\UnaryInfC{$\vdash\Theta : \Gamma\Uparrow P$}	
%				\UnaryInfC{$\vdash\Theta : \Gamma\Downarrow P$}
%			\end{prooftree}
%			
%		}
%	\end{multicols}
	
	
\end{tcolorbox}

%We can prove this cut in two parts. 
%
%{\footnotesize
%	\begin{multicols}{2}[\textbf{Initial Rules} : $\Theta$ is unbounded]
%		{
%	\begin{prooftree}
%	\AxiomC{$[\Pi_1]$}
%	\noLine
%	\UnaryInfC{$\vdash\Lambda,\Theta_1,H_i : \Gamma\Downarrow F$}
%	\AxiomC{$[\Pi_2]$}
%	\noLine
%	\UnaryInfC{$\vdash\Lambda : \cdot \Downarrow \nbang{i} H^{\bot}$}
%	\RightLabel{[$\Downarrow$CC]}
%	\BinaryInfC{$\vdash\Lambda,\Theta_1, : \Gamma\Uparrow F$}
%	\end{prooftree}
%	
%		\begin{prooftree}		
%	\AxiomC{$[\Pi_1]$}
%\noLine
%\UnaryInfC{$\vdash\Lambda,\Theta_1,H_i : \Gamma\Downarrow F$}
%\AxiomC{$[\Pi_2]$}
%\noLine
%\UnaryInfC{$\vdash\Lambda,\Theta_2 : \cdot \Downarrow \nbang{i} H^{\bot}$}
%\RightLabel{[$\Downarrow$CC]}
%\BinaryInfC{$\vdash\Lambda,\Theta_1,\Theta_2 : \Gamma\Uparrow F$}
%		\end{prooftree}
%}
%\end{multicols}
%}
%\vspace{0.4cm}
%\hrule
	
{\footnotesize
		\vspace{0.3cm}
\begin{minipage}{0.4\textwidth}
	\begin{prooftree}
		\AxiomC{}
		\UnaryInf$\fCenter\Kcal_1,\cutF_i : X^{+}\Downarrow X^{-}$
		\AxiomC{}
		\noLine
		\UnaryInfC{$\vdash\Kcal_2 : \cdot \Downarrow \nbang{i} \cutF^{\bot}$}
		\BinaryInfC{$\vdash\Kcal : X^{+}\Uparrow X^{-}$}
	\end{prooftree}
\end{minipage}
\begin{minipage}{0.1\textwidth}
	\begin{center}
		$\rightsquigarrow$
	\end{center}
\end{minipage}
\begin{minipage}{0.3\textwidth}
	\begin{prooftree}		
		\AxiomC{}
		\UnaryInf$\fCenter\Kcal : X^{+}\Downarrow X^{-}$
		\UnaryInf$\fCenter\Kcal : X^{+}\Uparrow X^{-}$
	\end{prooftree}
\end{minipage}
\vspace{0.3cm}

	\vspace{0.3cm}
\begin{minipage}{0.4\textwidth}
	\begin{prooftree}
		\AxiomC{}
		\UnaryInf$\fCenter\Kcal_1,X_a^{+},\cutF_i : \cdot\Downarrow X^{-}$
		\AxiomC{}
		\noLine
		\UnaryInfC{$\vdash\Kcal_2 : \cdot \Downarrow \nbang{i} \cutF^{\bot}$}
		\BinaryInfC{$\vdash\Kcal,X_a^{+} : \cdot\Uparrow X^{-}$}
	\end{prooftree}
\end{minipage}
\begin{minipage}{0.1\textwidth}
	\begin{center}
		$\rightsquigarrow$
	\end{center}
\end{minipage}
\begin{minipage}{0.3\textwidth}
	\begin{prooftree}		
	\AxiomC{}
	\UnaryInf$\fCenter\Kcal,X_a^{+} :  \cdot\Downarrow X^{-}$
	\UnaryInf$\fCenter\Kcal,X_a^{+} :  \cdot\Uparrow X^{-}$
\end{prooftree}
\end{minipage}
\vspace{0.3cm}

	\vspace{0.3cm}
\begin{minipage}{0.4\textwidth}
	\begin{prooftree}
		\AxiomC{}
		\UnaryInf$\fCenter\Kcal_1,X_a^{+},\cutF_i : \cdot\Downarrow X^{-}$
		\AxiomC{}
		\noLine
		\UnaryInfC{$\vdash\Kcal_2,X_a^{+} : \cdot \Downarrow \nbang{i} \cutF^{\bot}$}
		\BinaryInfC{$\vdash\Kcal,X_a^{+} : \cdot\Uparrow X^{-}$}
	\end{prooftree}
\end{minipage}
\begin{minipage}{0.1\textwidth}
	\begin{center}
		$\rightsquigarrow$
	\end{center}
\end{minipage}
\begin{minipage}{0.3\textwidth}
	\begin{prooftree}		
		\AxiomC{}
		\UnaryInf$\fCenter\Kcal,X_a^{+} :  \cdot\Downarrow X^{-}$
		\UnaryInf$\fCenter\Kcal,X_a^{+} :  \cdot\Uparrow X^{-}$
	\end{prooftree}
\end{minipage}
\vspace{0.3cm}

	\vspace{0.3cm}
\begin{minipage}{0.4\textwidth}
	\begin{prooftree}
		\AxiomC{}
		\UnaryInf$\fCenter\Kcal_1,X_i^{+}: \cdot\Downarrow X^{-}$
		\AxiomC{}
		\noLine
		\UnaryInfC{$\vdash\Kcal_2 : \cdot \Downarrow \nbang{i} X^{-}$}
		\BinaryInfC{$\vdash\Kcal : \cdot\Uparrow X^{-}$}
	\end{prooftree}
\end{minipage}
\begin{minipage}{0.1\textwidth}
	\begin{center}
		$\rightsquigarrow$
	\end{center}
\end{minipage}
\begin{minipage}{0.3\textwidth}
	\begin{prooftree}		
		\AxiomC{}
		\noLine
		\UnaryInfC{$\vdash\Kcal : \cdot \Downarrow \nbang{i} X^{-}$}
		\UnaryInfC{$\fCenter\Kcal :  \cdot\Uparrow X^{-}$}
	\end{prooftree}
\end{minipage}
\vspace{0.3cm}

Note above that $\{U,T\}\subset\mathcal{U}(i)$: then $\vdash\Kcal : \cdot \Downarrow \nbang{i} P$ implies {$\fCenter\Kcal : \cdot\Uparrow P$, for all $P$. In the case below, $i$ is linear. 
	
%	Since $T \in \mathcal{U}(i)$, we have $\Upsilon_+\equiv\Upsilon$

	\vspace{0.3cm}
\begin{minipage}{0.4\textwidth}
	\begin{prooftree}
		\AxiomC{}
		\UnaryInf$\fCenter\Kcal_1,X_i^{+}: \cdot\Downarrow X^{-}$
		\AxiomC{}
		\noLine
		\UnaryInfC{$\vdash\Upsilon,\Sigma^u_{c} : \cdot \Uparrow X^{-},\mathcal{F}(\Sigma^l)$}
		\UnaryInfC{$\vdash\Kcal_2 : \cdot \Downarrow \nbang{i} X^{-}$}
		\BinaryInfC{$\vdash\Kcal : \cdot\Uparrow X^{-}$}
	\end{prooftree}
\end{minipage}
\begin{minipage}{0.1\textwidth}
	\begin{center}
		$\rightsquigarrow$
	\end{center}
\end{minipage}
\begin{minipage}{0.3\textwidth}
	\begin{prooftree}		
		\AxiomC{}
		\noLine
	\UnaryInfC{$\fCenter\Upsilon,\Sigma^u_{c} :  \cdot\Uparrow X^{-},\mathcal{F}(\Sigma^l)$}
	\RightLabel{$\W$}
		\UnaryInfC{$\fCenter\Kcal,\Sigma^u_{c} :  \cdot\Uparrow X^{-},\mathcal{F}(\Sigma^l)$}
			\RightLabel{$\A_l$, $\C_c$}
		\UnaryInfC{$\fCenter\Kcal :  \cdot\Uparrow X^{-}$}
	\end{prooftree}
\end{minipage}
\vspace{0.3cm}


	
		\vspace{0.3cm}
	\begin{minipage}{0.4\textwidth}
		\begin{prooftree}
			\AxiomC{}
			\noLine
			\UnaryInf$\fCenter\Kcal_1,\cutF_i : \cdot\Downarrow 1$
			\AxiomC{}
			\noLine
			\UnaryInfC{$\vdash\Kcal_2 : \cdot \Downarrow \nbang{i} \cutF^{-}$}
		
			\BinaryInfC{$\vdash\Kcal : \cdot\Uparrow 1$}
		\end{prooftree}
	\end{minipage}
	\begin{minipage}{0.1\textwidth}
		\begin{center}
			$\rightsquigarrow$
		\end{center}
	\end{minipage}
	\begin{minipage}{0.3\textwidth}
		\begin{prooftree}		
			\AxiomC{}
			\UnaryInfC{$\vdash\Kcal : \cdot\Downarrow 1$}
			\UnaryInfC{$\vdash\Kcal : \cdot\Uparrow 1$}
		\end{prooftree}
	\end{minipage}
	\vspace{0.3cm}
		
	\vspace{0.3cm}
	\begin{minipage}{0.4\textwidth}
		\begin{prooftree}
			\AxiomC{[$\Sigma_1$]}
			\noLine
			\UnaryInfC{$\vdash\Kcal'_1,\cutF_i : \Gamma_P\Downarrow P$}
			\AxiomC{[$\Sigma_2$]}
			\noLine
			\UnaryInfC{$\vdash\Kcal''_1,\cutF_i : \Gamma_Q\Downarrow Q$}
			\BinaryInfC{$\vdash\Kcal_1,\cutF_i : \Gamma\Downarrow P\otimes Q$}
			\AxiomC{}
			\noLine
			\UnaryInfC{$\vdash\Kcal_2 : \cdot \Downarrow \nbang{i} \cutF^{\bot}$}
			\BinaryInfC{$\vdash\Kcal: \Gamma\Uparrow P\otimes Q$}
		\end{prooftree}
	\end{minipage}

	\begin{minipage}{0.5\textwidth}
		\begin{center}
			$\rightsquigarrow$
		\end{center}
	\end{minipage}
	\begin{minipage}{0.4\textwidth}
		\begin{prooftree}
			\AxiomC{}
			\noLine
			\UnaryInfC{$\Sigma_1$}
			\AxiomC{}
			\noLine
			\UnaryInfC{$\Pi_2$}
			\BinaryInfC{$\vdash\Kcal' : \Gamma_P\Uparrow P$}
			
			\AxiomC{}
			\noLine
			\UnaryInfC{$\Sigma_2$}
			\AxiomC{}
			\noLine
			\UnaryInfC{$\Pi_2$}
			\BinaryInfC{$\vdash\Kcal'' : \Gamma_Q\Uparrow Q$}
			\RightLabel{$\mathcal{L}\otimes$} 
			\BinaryInfC{$\vdash\Kcal: \Gamma,P\otimes Q\Uparrow \cdot$}
			\UnaryInfC{$\vdash\Kcal : \Gamma\Uparrow P\otimes Q$}
		\end{prooftree}
	\end{minipage}
	\vspace{0.3cm}

		
\vspace{0.3cm}
\begin{minipage}{0.4\textwidth}
	\begin{prooftree}
		\AxiomC{[$\Sigma_1$]}
		\noLine
		\UnaryInfC{$\vdash\Kcal'_1,,\cutF_i : \Gamma_P\Downarrow P$}
		\AxiomC{[$\Sigma_2$]}
		\noLine
		\UnaryInfC{$\vdash\Kcal''_1 : \Gamma_Q\Downarrow Q$}
		\BinaryInfC{$\vdash\Kcal_1,\cutF_i : \Gamma\Downarrow P\otimes Q$}
		\AxiomC{}
		\noLine
		\UnaryInfC{$\vdash\Kcal_2 : \cdot \Downarrow \nbang{i} \cutF^{\bot}$}
		\BinaryInfC{$\vdash\Kcal: \Gamma\Uparrow P\otimes Q$}
	\end{prooftree}
\end{minipage}

\begin{minipage}{0.5\textwidth}
	\begin{center}
		$\rightsquigarrow$
	\end{center}
\end{minipage}
\begin{minipage}{0.4\textwidth}
	\begin{prooftree}
		\AxiomC{}
		\noLine
		\UnaryInfC{$\Sigma_1$}
		\AxiomC{}
		\noLine
		\UnaryInfC{$\Pi_2$}
		\BinaryInfC{$\vdash\Kcal': \Gamma_P\Uparrow P$}
		
		\AxiomC{}
		\noLine
		\UnaryInfC{$\vdash\Kcal''_1 : \Gamma_Q\Downarrow Q$}
		\UnaryInfC{$\vdash\Kcal''_1 : \Gamma_Q\Uparrow Q$}
		\RightLabel{$\mathcal{L}\otimes$} 
		\BinaryInfC{$\vdash\Kcal: \Gamma,P\otimes Q\Uparrow \cdot$}
		\UnaryInfC{$\vdash\Kcal : \Gamma\Uparrow P\otimes Q$}
	\end{prooftree}
\end{minipage}
\vspace{0.3cm}


Same reasoning for $\oplus$ and $\exists$, see below.

	\vspace{0.3cm}
	\begin{minipage}{0.4\textwidth}
		\begin{prooftree}
			\AxiomC{}
			\noLine
			\UnaryInf$\fCenter\Kcal_1,\cutF_i : \Gamma\Downarrow P$
			\UnaryInf$\fCenter\Kcal_1,\cutF_i : \Gamma\Downarrow P\oplus Q$
			\AxiomC{}
			\noLine
			\UnaryInfC{$\Pi_2$}
			\BinaryInfC{$\vdash\Kcal : \Gamma\Uparrow P\oplus Q$}
		\end{prooftree}
	\end{minipage}
	\begin{minipage}{0.1\textwidth}
		\begin{center}
			$\rightsquigarrow$
		\end{center}
	\end{minipage}
	\begin{minipage}{0.3\textwidth}
		\begin{prooftree}
			
					\AxiomC{}
		\noLine
		\UnaryInf$\fCenter\Kcal_1,\cutF_i : \Gamma\Downarrow P$
		\AxiomC{}
		\noLine
		\UnaryInfC{$\Pi_2$}
			
			\BinaryInfC{$\vdash\Kcal : \Gamma\Uparrow P$}
			\RightLabel{$\mathcal{L}\oplus$} 
			\UnaryInfC{$\fCenter\Kcal : \Gamma,P\oplus Q\Uparrow \cdot$}
			\UnaryInfC{$\fCenter\Kcal : \Gamma\Uparrow P\oplus Q$}
		\end{prooftree}
	\end{minipage}
	\vspace{0.3cm}
	

	\vspace{0.3cm}
	\begin{minipage}{0.4\textwidth}
		\begin{prooftree}
			\AxiomC{}
			\noLine
			\UnaryInf$\fCenter\Kcal_1,\cutF_i : \Gamma\Downarrow P[c/x]$
			\UnaryInf$\fCenter\Kcal_1,\cutF_i : \Gamma\Downarrow \exists x.P$
			\AxiomC{}
			\noLine
			\UnaryInfC{$\Pi_2$}
			\BinaryInfC{$\vdash\Kcal : \Gamma\Uparrow \exists x.P $}
		\end{prooftree}
	\end{minipage}
	\begin{minipage}{0.1\textwidth}
		\begin{center}
			$\rightsquigarrow$
		\end{center}
	\end{minipage}
	\begin{minipage}{0.3\textwidth}
		\begin{prooftree}
				
			\AxiomC{}
			\noLine
			\UnaryInf$\fCenter\Kcal_1,\cutF_i : \Gamma\Downarrow P[c/x]$
			\AxiomC{}
			\noLine
			\UnaryInfC{$\Pi_2$}
			
			\BinaryInfC{$\vdash\Kcal : \Gamma\Uparrow  P[c/x]$}
			\RightLabel{$\mathcal{L}\exists$} 
			\UnaryInfC{$\fCenter\Kcal : \Gamma,\exists x.P\Uparrow \cdot$}
			\UnaryInfC{$\fCenter\Kcal : \Gamma\Uparrow \exists x.P$}
			
		\end{prooftree}
	\end{minipage}
	\vspace{0.3cm}
	

			
\begin{tcolorbox}
			\vspace{0.3cm}
\begin{minipage}{0.4\textwidth}
	\begin{prooftree}
		\AxiomC{}
		\noLine
		\UnaryInf$\fCenter\Kcal_1,\cutF_i : \Gamma\Uparrow P$
		\UnaryInf$\fCenter\Kcal_1,\cutF_i : \Gamma\Downarrow P$
		\AxiomC{}
		\noLine
		\UnaryInfC{$\foc{\Kcal_2}{\cdot}{\nbang{i} \cutF^\perp}$}
		\BinaryInfC{$\vdash\Kcal : \Gamma\Uparrow P $}
	\end{prooftree}
\end{minipage}
\begin{minipage}{0.1\textwidth}
	\begin{center}
		$\rightsquigarrow$
	\end{center}
\end{minipage}
\begin{minipage}{0.3\textwidth}
	\begin{prooftree}
		
		\AxiomC{}
		\noLine
		\UnaryInfC{$\vdash\Kcal_1,\cutF_i : \Gamma\Uparrow P$}
		
		\AxiomC{}
		\noLine
		\UnaryInfC{$\foc{\Kcal_2}{\cdot}{\nbang{i} \cutF^\perp}$}			
		\RightLabel{$\cut_3$}
		\BinaryInfC{$\fCenter\Kcal : \Gamma\Uparrow P$}
	\end{prooftree}
\end{minipage}
\vspace{0.3cm}	
	
	
	\begin{minipage}{0.4\textwidth}
		\begin{prooftree}
			\AxiomC{}
			\noLine
			\UnaryInf$\fCenter\Kcal_1,\cutF_i : \cdot\Uparrow P$
			\UnaryInf$\fCenter\Kcal_1,\cutF_i : \cdot\Downarrow \nbang{c} P$
			\AxiomC{}
			\noLine
			\UnaryInfC{$\foc{\Kcal_2}{\cdot}{\nbang{i} \cutF^\perp}$}
			\BinaryInfC{$\vdash\Kcal : \cdot\Uparrow \nbang{c} P $}
		\end{prooftree}
	\end{minipage}
	\begin{minipage}{0.1\textwidth}
		\begin{center}
			$\rightsquigarrow$
		\end{center}
	\end{minipage}
	\begin{minipage}{0.3\textwidth}
		\begin{prooftree}
			
			\AxiomC{}
			\noLine
			\UnaryInfC{$\vdash\Kcal_1,\cutF_i : \cdot\Uparrow P$}
			
			\AxiomC{}
			\noLine
			\UnaryInfC{$\foc{\Kcal_2}{\cdot}{\nbang{i} \cutF^\perp}$}			
			\RightLabel{$\cut_3$}
			\BinaryInfC{$\fCenter\Kcal : \cdot\Uparrow P$}
			\UnaryInfC{$\fCenter\Kcal : \cdot\Downarrow \nbang{c} P$}
			\doubleLine
			\UnaryInfC{$\fCenter\Kcal : \cdot\Uparrow \nbang{c} P$}
		\end{prooftree}
	\end{minipage}
	\vspace{0.3cm}
	
\end{tcolorbox}
\vspace{0.3cm}



{	\footnotesize
	\vspace{0.3cm}	
	\begin{tcolorbox}
		$a\preceq i$ and $4 \in \mathcal{U}(i) $
		
		\begin{minipage}{0.5\textwidth}
			\begin{prooftree}
				\AxiomC{}
				\noLine
				\UnaryInfC{$\fCenter \Upsilon_+,H_{i+},\Sigma^u_{lc} ; \cdot\Uparrow P,\mathcal{F}(\Sigma^l)$}
				\UnaryInfC{$\fCenter \Lambda,\Theta_1, H_i ; \cdot\Downarrow \nbang{a} P$}
				\AxiomC{}
				\noLine
				\UnaryInfC{$\fCenter \Upsilon^{\prime}_+; \cdot\Uparrow H^{\bot}$}
				\UnaryInfC{$\fCenter \Lambda,\Theta_2; \cdot\Downarrow \nbang{i} H^{\bot}$}
				\BinaryInfC{$\fCenter \Lambda,\Theta_1,\Theta_2; \cdot\Uparrow \nbang{a} P $}
			\end{prooftree}
		\end{minipage}

	\begin{minipage}{0.2\textwidth}
		\begin{center}
			$\rightsquigarrow$
		\end{center}
	\end{minipage}	
		\begin{minipage}{0.4\textwidth}
			\begin{prooftree}
				
				\AxiomC{}
				\noLine
				\UnaryInfC{$\fCenter \Upsilon_+,\Sigma^u_{lc} ; \cdot\Uparrow P,\mathcal{F}(\Sigma)$}	
				\RightLabel{[$W$]}
				\UnaryInfC{$\fCenter (\Upsilon^{\prime}_+)^u,\Upsilon_+,\Sigma^u_{lc} ; \cdot\Uparrow P,\mathcal{F}(\Sigma)$}	
				
				
				\AxiomC{}
				\noLine
				\UnaryInfC{$\fCenter\Upsilon^{\prime}_+; \cdot\Downarrow \nbang{i+} H^{\bot}$}
					\RightLabel{[$W$]}
				\UnaryInfC{$\fCenter\Upsilon^{\prime}_+,(\Upsilon_+)^u,\Sigma^u_{lc}; \cdot\Downarrow \nbang{i+} H^{\bot}$}
				
				\RightLabel{[$\Uparrow$CC]}
				\BinaryInfC{$\fCenter \Upsilon_+,\Upsilon^{\prime}_+,\Sigma^u_{lc}; \cdot\Uparrow P,\mathcal{F}(\Sigma_1)$}
				\doubleLine
				\RightLabel{C3}
				\UnaryInfC{$\fCenter \Lambda,\Theta_1,\Theta_2 ; \cdot\Downarrow \nbang{a} P$}
				\UnaryInfC{$\fCenter \Lambda,\Theta_1,\Theta_2 ; \nbang{a} P\Uparrow \cdot$}
				\UnaryInfC{$\fCenter \Lambda,\Theta_1,\Theta_2 ; \cdot\Uparrow \nbang{a} P$}
			\end{prooftree}
		\end{minipage}
	\end{tcolorbox}
	\vspace{0.3cm}	
	
	\vspace{0.3cm}	
	\begin{tcolorbox}
		$a\preceq i$ and $4 \notin \mathcal{U}(i) $
		
		\begin{minipage}{0.5\textwidth}
			\begin{prooftree}
				\AxiomC{}
				\noLine
				\UnaryInfC{$\fCenter \Upsilon_{+},\Sigma^u_{lc},H_{lc} ; \cdot\Uparrow P,\mathcal{F}(\Sigma)$}
				\UnaryInfC{$\fCenter \Lambda,\Theta_1, H_i ; \cdot\Downarrow \nbang{a} P$}
				\AxiomC{}
				\noLine
				\UnaryInfC{$\fCenter \Lambda; \cdot\Uparrow H^{\bot}$}
				\UnaryInfC{$\fCenter \Lambda; \cdot\Downarrow \nbang{i} H^{\bot}$}
				\BinaryInfC{$\fCenter \Lambda,\Theta_1 ; \cdot\Uparrow \nbang{a} P $}
			\end{prooftree}
		\end{minipage}
		\vspace{0.4cm}	
		
		\begin{minipage}{0.4\textwidth}
			\begin{prooftree}
				
				\AxiomC{}
				\noLine
				\UnaryInfC{$\fCenter \Upsilon_{+},\Sigma^u_{lc},H_{lc} ; \cdot\Uparrow P,\mathcal{F}(\Sigma)$}
				\RightLabel{[$W$]}	
				\UnaryInfC{$\fCenter \Lambda,\Upsilon_{+},\Sigma^u_{lc},H_{lc} ; \cdot\Uparrow P,\mathcal{F}(\Sigma)$}	
				
				
				\AxiomC{}
				\noLine
				\UnaryInfC{$\fCenter \Lambda ; \cdot\Downarrow \nbang{lc} H^{\bot}$}
				\RightLabel{[$W$]}
				\UnaryInfC{$\fCenter \Lambda,\Upsilon^u_{+},\Sigma^u_{lc}; \cdot\Downarrow \nbang{lc} H^{\bot}$}
				
				\RightLabel{[$\Uparrow$CC]}
				\BinaryInfC{$\fCenter \Lambda,\Upsilon_{+},\Sigma^u_{lc}; \cdot\Uparrow P,\mathcal{F}(\Sigma)$}
				\doubleLine
				\UnaryInfC{$\fCenter \Lambda,\Theta_1 ; \cdot\Downarrow \nbang{a} P$}
				\UnaryInfC{$\fCenter \Lambda,\Theta_1 ; \nbang{a} P\Uparrow \cdot$}
				\UnaryInfC{$\fCenter \Lambda,\Theta_1 ; \cdot\Uparrow \nbang{a} P$}
			\end{prooftree}
		\end{minipage}
		\vspace{0.3cm}
				
	\end{tcolorbox}
	\vspace{0.3cm}

		\vspace{0.3cm}	
		\begin{tcolorbox}
			$a\preceq i$ and $4 \notin \mathcal{U}(i) $
			
			\begin{minipage}{0.5\textwidth}
				\begin{prooftree}
					\AxiomC{}
					\noLine
					\UnaryInfC{$\fCenter \Upsilon_+,\Sigma^u_{lc} ; \cdot\Uparrow P,\mathcal{F}(\Sigma^l_1),H,\mathcal{F}(\Sigma^l_3)$}
					\UnaryInfC{$\fCenter \Lambda,\Theta_1, H_i ; \cdot\Downarrow \nbang{a} P$}
					\AxiomC{}
					\noLine
					\UnaryInfC{$\fCenter\Upsilon^{\prime}_+,(\Sigma^{\prime})^u_{lc}; \cdot\Uparrow H^{\bot},\mathcal{K}(\Sigma^l_2)$}
					\UnaryInfC{$\fCenter \Lambda,\Theta_2; \cdot\Downarrow \nbang{i} H^{\bot}$}
					\BinaryInfC{$\fCenter \Lambda,\Theta_1,\Theta_2 ; \cdot\Downarrow \nbang{a} P $}
				\end{prooftree}
			\end{minipage}
			\vspace{0.3cm}
			
			
			\vspace{0.3cm}
			\begin{minipage}{0.3\textwidth}
				\begin{prooftree}
					\AxiomC{}
					\noLine
					\UnaryInfC{$\fCenter\Upsilon_+,\Sigma^u_{lc} ; \cdot\Uparrow P,\mathcal{F}(\Sigma_1),H,\mathcal{F}(\Sigma_3)$}
					\RightLabel{[$W$]}
					\UnaryInfC{$\fCenter\Upsilon_+,\Sigma^u_{lc},(\Upsilon^{\prime})^u_+,(\Sigma^{\prime})^u_{lc} ; \cdot\Uparrow P,\mathcal{F}(\Sigma_1),H,\mathcal{F}(\Sigma_3)$}		
					
					
					\AxiomC{}
					\noLine
					\UnaryInfC{$\fCenter\Upsilon^{\prime}_+,(\Sigma^{\prime})^u_{lc} ; \cdot\Uparrow H^{\bot},\mathcal{K}(\Sigma_2)$}
					\RightLabel{[$W$]}
					\UnaryInfC{$\fCenter\Upsilon^u_+,\Sigma^u_{lc},\Upsilon^{\prime}_+,(\Sigma^{\prime})^u_{lc} ; \cdot\Uparrow H^{\bot},\mathcal{K}(\Sigma_2)$}
					
					\RightLabel{[$\Uparrow$C*]}
					\BinaryInfC{$\fCenter\fCenter \Upsilon_+,\Sigma^u_{lc},\Upsilon^{\prime}_+,(\Sigma^{\prime})^u_{lc} ; \cdot\Uparrow P,\mathcal{F}(\Sigma_1),\mathcal{F}(\Sigma_3),\mathcal{F}(\Sigma_2)$}
					\doubleLine
					\UnaryInfC{$\fCenter \Lambda,\Theta_1,\Theta_2 ; \cdot\Downarrow \nbang{a} P$}
					\UnaryInfC{$\fCenter \Lambda,\Theta_1,\Theta_2 ; \nbang{a} P\Uparrow \cdot$}
					\UnaryInfC{$\fCenter \Lambda,\Theta_1,\Theta_2 ; \cdot\Uparrow \nbang{a} P$}
				\end{prooftree}
			\end{minipage}
			\vspace{0.3cm}
			
			
			
		\end{tcolorbox}
		\vspace{0.3cm}
	}
	

\newpage
\section{Elimination of $\cut_5$}
{\footnotesize	
	\vspace{0.3cm}	
	
\begin{center}
		$	\infer[\cut_5]{\unfoc{\Kcal}{\Gamma,\Delta}{L_1,L_2,L_3}}{
		\deduce{\unfoc{\Kcal_1}{\Gamma}{L_1,H,L_2}}{}
		&
		\deduce{\unfoc{\Kcal_2}{\Delta}{H^\perp,L_3}}{}
	}$

\end{center}			
	\vspace{0.3cm}	
}

{\footnotesize
	
	\vspace{0.3cm}
	\begin{minipage}{0.4\textwidth}
		\begin{prooftree}
			\AxiomC{}
			\noLine
			\UnaryInf$\fCenter\Kcal_1 : \Gamma \Uparrow L_1,\cutF,L_2$
			\UnaryInf$\fCenter \Kcal_1 : \Gamma \Uparrow  \bot, L_1,\cutF,L_2$
			\AxiomC{}
			\noLine
			\UnaryInfC{$\Pi_2$}
			\BinaryInfC{$\vdash \Kcal : \Gamma,\Delta\Uparrow \bot, L_1,L_2,L_3$}
		\end{prooftree}
	\end{minipage}
	\begin{minipage}{0.1\textwidth}
		\begin{center}
			$\rightsquigarrow$
		\end{center}
	\end{minipage}
	\begin{minipage}{0.3\textwidth}
		\begin{prooftree}
			\AxiomC{}
			\noLine
			\UnaryInfC{$\vdash \Kcal_1 : \Gamma \Uparrow L_1,\cutF,L_2$}
			\AxiomC{}
			\noLine
			\UnaryInfC{$\Pi_2$}
			\BinaryInf$\fCenter \Kcal : \Gamma,\Delta\Uparrow L_1,L_2,L_3$
			\UnaryInf$\fCenter \Kcal : \Gamma,\Delta\Uparrow \bot, L_1,L_2,L_3$
		\end{prooftree}
	\end{minipage}
	\vspace{0.3cm}
	
	
	\vspace{0.3cm}
	\begin{minipage}{0.4\textwidth}
		\begin{prooftree}
			\AxiomC{}
			\noLine
			\UnaryInf$\fCenter \Kcal_1 : \Gamma \Uparrow  P, Q, L_1,\cutF,L_2$
			\UnaryInf$\fCenter \Kcal_1 : \Gamma \Uparrow  P\parr Q, L_1,\cutF,L_2$
			\AxiomC{}
			\noLine
			\UnaryInfC{$\Pi_2$}
			\BinaryInfC{$\vdash \Kcal : \Gamma,\Delta\Uparrow P\parr Q, L_1,L_2,L_3$}
		\end{prooftree}
	\end{minipage}
	\begin{minipage}{0.1\textwidth}
		\begin{center}
			$\rightsquigarrow$
		\end{center}
	\end{minipage}
	\begin{minipage}{0.3\textwidth}
		\begin{prooftree}
			\AxiomC{}
			\noLine
			\UnaryInf$\fCenter \Kcal_1 :  \Gamma \Uparrow  P, Q, L_1,\cutF,L_2$
			\AxiomC{}
			\noLine
			\UnaryInfC{$\Pi_2$}
			\BinaryInf$\fCenter \Kcal : \Gamma,\Delta\Uparrow P, Q, L_1,L_2,L_3$
			\UnaryInf$\fCenter \Kcal : \Gamma,\Delta\Uparrow P\parr Q, L_1,L_2,L_3  $
		\end{prooftree}
	\end{minipage}
	\vspace{0.3cm}
	
	\vspace{0.3cm}	
	\begin{minipage}{0.4\textwidth}
		\begin{prooftree}
			\AxiomC{}
			\noLine
			\UnaryInfC{$\vdash \Kcal_1 :\Gamma \Uparrow  P, L_1,\cutF,L_2$}
			\AxiomC{}
			\noLine
			\UnaryInfC{$\vdash \Kcal_1 : \Gamma \Uparrow  Q, L_1,\cutF,L_2$}
			\BinaryInf$\fCenter \Kcal_1 : \Gamma \Uparrow  P\with Q, L_1,\cutF,L_2$
			\AxiomC{}
			\noLine
			\UnaryInfC{$\Pi_2$}
			\BinaryInfC{$\vdash \Kcal : \Gamma,\Delta\Uparrow P\with Q, L_1,L_2,L_3$}
		\end{prooftree}
	\end{minipage}
	
	\begin{minipage}{0.3\textwidth}
		\begin{center}
			$\rightsquigarrow$
		\end{center}
	\end{minipage}
	\begin{minipage}{0.3\textwidth}
		\begin{prooftree}
			\AxiomC{}
			\noLine
			\UnaryInf$\fCenter \Kcal_1 :  \Gamma \Uparrow  P, L_1,\cutF,L_2$
			\AxiomC{}
			\noLine
			\UnaryInfC{$\Pi_2$}
			\BinaryInfC{$\vdash \Kcal : \Gamma,\Delta \Uparrow  P, L_1,L_2,L_3$}
			\AxiomC{}
			\noLine
			\UnaryInf$\fCenter \Kcal_1 : \Gamma \Uparrow Q, L_1,\cutF,L_2$
			\AxiomC{}
			\noLine
			\UnaryInfC{$\Pi_2$}
			\BinaryInfC{$\vdash \Kcal : \Gamma,\Delta \Uparrow  Q, L_1,L_2,L_3$}
			\BinaryInfC{$\vdash \Kcal : \Gamma,\Delta\Uparrow P\with Q, L_1,L_2,L_3$}
		\end{prooftree}
	\end{minipage} 
	\vspace{0.3cm}
	
	\vspace{0.3cm}
	\begin{minipage}{0.4\textwidth}
		\begin{prooftree}
			\AxiomC{}
			\noLine
			\UnaryInf$\fCenter \Kcal_1 : \Gamma \Uparrow  P[c/x],L_1,\cutF,L_2$
			\UnaryInf$\fCenter \Kcal_1 : \Gamma \Uparrow  \forall xP,L_1,\cutF,L_2$
			\AxiomC{}
			\noLine
			\UnaryInfC{$\Pi_2$}
			\BinaryInfC{$\vdash \Kcal : \Gamma,\Delta\Uparrow \forall xP,L_1,L_2,L_3$}
		\end{prooftree}
	\end{minipage}
	\begin{minipage}{0.1\textwidth}
		\begin{center}
			$\rightsquigarrow$
		\end{center}
	\end{minipage}
	\begin{minipage}{0.3\textwidth}
		\begin{prooftree}
			\AxiomC{}
			\noLine
			\UnaryInfC{$\vdash \Kcal_1 : \Gamma \Uparrow P[c/x] ,L_1,\cutF,L_2$}
			\AxiomC{}
			\noLine
			\UnaryInfC{$\Pi_2$}
			\BinaryInfC{$\vdash \Kcal : \Gamma,\Delta\Uparrow  P[c/x],L_1,L_2,L_3$}
			\UnaryInfC{$\vdash \Kcal : \Gamma,\Delta\Uparrow \forall xP,L_1,L_2,L_3$}
		\end{prooftree}
	\end{minipage} 
	\vspace{0.3cm}
	
	\vspace{0.3cm}	
	\begin{minipage}{0.4\textwidth}
		\begin{prooftree}
			\AxiomC{}
			\noLine
			\UnaryInf$\fCenter\Kcal_1,P_i: \Gamma \Uparrow L_1,\cutF,L_2$
			\UnaryInf$\fCenter\Kcal_1:  \Gamma \Uparrow \nquest{i} P,L_1,\cutF,L_2$
			\AxiomC{}
			\noLine
			\UnaryInfC{$\Pi_2$}
			\BinaryInfC{$\vdash \Kcal : \Gamma,\Delta\Uparrow \nquest{i} P, L_1,L_2,L_3$}
		\end{prooftree}
	\end{minipage}
	\begin{minipage}{0.1\textwidth}
		\begin{center}
			$\rightsquigarrow$
		\end{center}
	\end{minipage}
	\begin{minipage}{0.3\textwidth}
		\begin{prooftree}
			\AxiomC{}
			\noLine
			\UnaryInf$\fCenter\Kcal_1,P_i: \Gamma \Uparrow L_1,\cutF,L_2$
			\AxiomC{}
			\noLine
			\UnaryInfC{$\Pi_2$}
			\BinaryInf$\fCenter\Kcal,P_i: \Gamma,\Delta \Uparrow L_1,L_2,L_3$
			\UnaryInf$\fCenter \Kcal : \Gamma,\Delta\Uparrow\nquest{i} P, L_1,L_2,L_3$
		\end{prooftree}
	\end{minipage}
	\vspace{0.3cm}
	
	\vspace{0.3cm}	
	\begin{minipage}{0.4\textwidth}
		\begin{prooftree}
			\AxiomC{}
			\noLine
			\UnaryInf$\fCenter\Kcal_1,P_i: \Gamma \Uparrow L_1,\cutF,L_2$
			\UnaryInf$\fCenter\Kcal_1: H, \Gamma \Uparrow \nquest{i} P, L_1,\cutF,L_2$
			\AxiomC{}
			\noLine
			\UnaryInfC{$\Pi_2$}
			\BinaryInfC{$\vdash\Kcal : \Gamma,\Delta\Uparrow \nquest{i} P, L_1,L_2,L_3$}
		\end{prooftree}
	\end{minipage}
	\begin{minipage}{0.1\textwidth}
		\begin{center}
			$\rightsquigarrow$
		\end{center}
	\end{minipage}
	\begin{minipage}{0.3\textwidth}
		\begin{prooftree}
			\AxiomC{}
			\noLine
			\UnaryInf$\fCenter\Kcal_1,P_i: \Gamma \Uparrow L_1,\cutF,L_2$
			\AxiomC{}
			\noLine
			\UnaryInfC{$\Pi_2$}
			\RightLabel{$\W$}
			\UnaryInfC{$\fCenter\Kcal_2,P_i: \Delta \Uparrow  H^{\bot},V$}
			\BinaryInf$\fCenter\Kcal,P_i: \Gamma,\Delta \Uparrow L_1,L_2,L_3$
			\UnaryInf$\fCenter\Kcal : \Gamma,\Delta\Uparrow\nquest{i} P, L_1,L_2,L_3$
		\end{prooftree}
	\end{minipage}
	\vspace{0.3cm}
	
	\vspace{0.3cm}	
	\begin{minipage}{0.4\textwidth}
		\begin{prooftree}
			\AxiomC{}
			\noLine
			\UnaryInf$\fCenter\Kcal_1 : P, \Gamma \Uparrow L_1,\cutF,L_2$
			\UnaryInf$\fCenter\Kcal_1 :  \Gamma \Uparrow P, L_1,\cutF,L_2$
			\AxiomC{}
			\noLine
			\UnaryInfC{$\Pi_2$}
			\BinaryInfC{$\vdash \Kcal :\Gamma,\Delta\Uparrow P, L_1,L_2,L_3$}
		\end{prooftree}
	\end{minipage}
	\begin{minipage}{0.1\textwidth}
		\begin{center}
			$\rightsquigarrow$
		\end{center}
	\end{minipage}
	\begin{minipage}{0.3\textwidth}
		\begin{prooftree}
			\AxiomC{}
			\noLine
			\UnaryInf$\fCenter \Kcal_1 : P, \Gamma \Uparrow  L_1,\cutF,L_2$
			\AxiomC{}
			\noLine
			\UnaryInfC{$\Pi_2$}
			\BinaryInf$\fCenter \Kcal : P, \Gamma,\Delta \Uparrow  L_1,L_2,L_3$
			\UnaryInf$\fCenter \Kcal : \Gamma,\Delta\Uparrow P, L_1,L_2,L_3$
		\end{prooftree}
	\end{minipage}
	\vspace{0.3cm}
	
	
	\vspace{0.3cm}
	\begin{tcolorbox}
		\begin{minipage}{0.4\textwidth}
			\begin{prooftree}
				\AxiomC{}
				\noLine
				\UnaryInf$\fCenter\Kcal_1 : \Gamma,P\Uparrow L_2$
				
				\UnaryInf$\fCenter\Kcal_1 : \Gamma\Uparrow P,L_2$
				\AxiomC{}
				\noLine
				\UnaryInf$\fCenter\Kcal_2 : \Delta\Uparrow P^{\bot},L_3$
				
				\BinaryInfC{$\vdash \Kcal : \Gamma,\Delta\Uparrow L_2,L_3$}
			\end{prooftree}
		\end{minipage}
		\begin{minipage}{0.1\textwidth}
			\begin{center}
				$\rightsquigarrow$
			\end{center}
		\end{minipage}
		\begin{minipage}{0.3\textwidth}
			\begin{prooftree}
				\AxiomC{}
				\noLine
				\UnaryInf$\fCenter\Kcal_1 :  P,\Gamma\Downarrow L_2$
				\AxiomC{}
				\noLine
				\UnaryInfC{$\fCenter \Kcal_2 : \Delta\Uparrow P^{\bot},L_3$}
				\RightLabel{$\cut_1$}
				\BinaryInf$\fCenter \Kcal : \Gamma,\Delta\Uparrow L_2,L_3$
			\end{prooftree}
		\end{minipage}
		\vspace{0.6cm}	

		\begin{minipage}{0.4\textwidth}
	\begin{prooftree}
		\AxiomC{}
		\noLine
		\UnaryInf$\fCenter\Kcal_1 : \Gamma\Uparrow P,L_2$
		\AxiomC{}
		\noLine
			\UnaryInf$\fCenter\Kcal_2 : \Delta,P^{\bot}\Uparrow L_3$
		
		\UnaryInf$\fCenter\Kcal_2 : \Delta\Uparrow P^{\bot},L_3$
		
		\BinaryInfC{$\vdash \Kcal : \Gamma,\Delta\Uparrow  L_2,L_3$}
	\end{prooftree}
\end{minipage}
\begin{minipage}{0.1\textwidth}
	\begin{center}
		$\rightsquigarrow$
	\end{center}
\end{minipage}
\begin{minipage}{0.3\textwidth}
	\begin{prooftree}
		\AxiomC{}
		\noLine
		\UnaryInf$\fCenter\Kcal_1 :  P^{\bot},\Delta\Downarrow L_3$
		\AxiomC{}
		\noLine
		\UnaryInfC{$\fCenter \Kcal_2 : \Gamma\Uparrow P,L_2$}
		\RightLabel{$\cut_1$}
		\BinaryInf$\fCenter \Kcal : \Gamma,\Delta\Uparrow L_3,L_2$
		\UnaryInf$\fCenter \Kcal : \Gamma,\Delta\Uparrow L_2,L_3$
	\end{prooftree}
\end{minipage}	
	\end{tcolorbox}
	
	\vspace{0.3cm}	



\end{document}
